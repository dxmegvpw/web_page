\documentclass[unicode,a4paper,10pt]{ltjsarticle}

\usepackage{luatexja-fontspec}
\setmainfont{TeX Gyre Termes}
\setmainjfont[BoldFont = IPAexGothic]{IPAexMincho}
\setmathrm{Latin Modern Roman}
% \setmainjfont{Noto Sans JP}

% ---Display \subsubsection at the Index
% \setcounter{tocdepth}{3}

% ---Setting about the geometry of the document----
% \usepackage{a4wide}
% \pagestyle{empty}

% ---Physics and Math Packages---
\usepackage{amssymb,amsfonts,amsthm,mathtools}
\usepackage{physics,braket,bm}

% ---underline---
\usepackage[normalem]{ulem}

% ---cancel---
\usepackage{cancel}

% --- surround the texts or equations
% \usepackage{fancybox,ascmac}

% ---settings of theorem environment---
% \usepackage{amsthm}
% \theoremstyle{definition}

% ---settings of proof environment---
% \renewcommand{\proofname}{\textbf{証明}}
% \renewcommand{\qedsymbol}{$\blacksquare$}

% ---Ignore the Warnings---
\usepackage{silence}
\WarningFilter{latexfont}{Some font shapes}
\WarningFilter{latexfont}{Font shape}
\ExplSyntaxOn
\msg_redirect_name:nnn{hooks}{generic-deprecated}{none}
\ExplSyntaxOff

% ---Insert the figure (If insert the `draft' at the option, the process becomes faster.)---
\usepackage{graphicx}
% \usepackage{subcaption}

% ----Add a link to a text---
\usepackage{url,hyperref}
\usepackage[dvipsnames,svgnames]{xcolor}
\hypersetup{colorlinks=true,citecolor=FireBrick,linkcolor=Navy,urlcolor=purple}
% ---refer `texdoc xcolor' at the command line---

% ---Tikz---
% \usepackage{tikz,pgf,pgfplots,circuitikz}
% \pgfplotsset{compat=1.15}
% \usetikzlibrary{intersections,arrows.meta,angles,calc,3d,decorations.pathmorphing}

% ---Add the section number to the equation, figure, and table number---
\makeatletter
   \renewcommand{\theequation}{\thesection.\arabic{equation}}
   \@addtoreset{equation}{section}
   
   \renewcommand{\thefigure}{\thesection.\arabic{figure}}
   \@addtoreset{figure}{section}
   
   \renewcommand{\thetable}{\thesection.\arabic{table}}
   \@addtoreset{table}{section}
\makeatother

% ---enumerate---
% \renewcommand{\labelenumi}{$\arabic{enumi}.$}
% \renewcommand{\labelenumii}{$(\arabic{enumii})$}

% ---Index---
% \usepackage{makeidx}
% \makeindex 

% ---Fonts---
% \renewcommand{\familydefault}{\sfdefault}

% ---footnotes---
% \renewcommand{\thefootnote}{$\ast$\arabic{footnote}}

% ---Title---
\title{高次元時空モデルと素粒子標準模型}
\author{宮根\ 一樹}
\date{2024年\ 3月}

\begin{document}

\maketitle

\begin{abstract}
   これは数物セミナー素粒子グループでのリレーセミナーの資料になります。私が現在所属している研究室\footnote{
      早稲田大学の安倍研究室といいます。ホームページは\href{http://www.hep.phys.waseda.ac.jp/index-j.html}{こちら}です。
   }は、4次元よりも大きい次元の時空を仮定する高次元時空モデルから素粒子標準模型を再現することを目標とした研究を主にしています。そこで、ここでは高次元時空モデルからどのようにして統一理論の見通しがつくのかについて、議論したいと思います。
\end{abstract}

\tableofcontents

\clearpage

\section{はじめに}


\clearpage

\section{高次元の場の理論とコンパクト化}

この章では、$4+n$次元のEinstein-Hilbert作用がコンパクト化によって、4次元では、一般相対論とYang-Mills理論になることを見ていこうと思います。


\subsection{5次元理論のコンパクト化}

まずは単純に5次元の時空を考えることにします。その座標は$z^{M}$で、$M=0,1,2,3,4$という値をとります。また、時空はミンコフスキーで計量は
\begin{equation}
   \eta_{MN}
   =
   \mathrm{diag}(-,+,+,+,+)
\end{equation}
です。この時空上の場の理論は、たとえば実スカラー場$\phi(z)$なら
\begin{equation}
   S
   =
   \int\dd^5 z\ \sqrt{-g}
   \left(  
      -\frac{1}{2}
      g^{MN}\partial_{M}\phi\partial_{N}\phi
   \right)
   \label{eqn:action_5d_real_scalar}
\end{equation}
のように4次元の理論と同じように議論することができます。

ここで、5番目の座標$z_{4}$が半径$a$の円周になったとしましょう。円周$S^{1}$となったときは$\theta\in[0,2\pi)$でパラメトライズするのが便利なので、以後は無次元のパラメターで$z_{4}(\theta)$と書けるとします。さらに
\begin{equation}
   \dd z_{4}
   =
   a\dd \theta
\end{equation}
という関係が成立するとすれば、$\dd z_{4}^2=a^2\dd \theta^2$より
\begin{equation}
   g_{\theta\theta}=a^2
   。
\end{equation}

上述のように、余剰空間に境界条件を課すことを\textbf{コンパクト化}といいます。ここで、5次元のミンコフスキー時空$M_{5}$がコンパクト化によって時空が4次元ミンコフスキーと円周の直積$M_{4}\times S^{1}$になるとすれば、その計量は
\begin{equation}
   g_{MN}
   =
   \begin{pmatrix}
      \eta_{\mu\nu} & 0 \\
      0 & a^2
   \end{pmatrix}
\end{equation}
です\footnote{
   ただし、$g_{\mu\theta}=g_{\theta\mu}=0$と4次元の部分と1次元の部分が完全に分離できているのは仮定です。
}。

このような時空の上で、実スカラー場$\phi(z)$の理論を考えてみましょう。今、$\theta$方向は円周$S^{1}$にコンパクト化されていることから、$\phi(z)$はフーリエ級数展開できます:
\begin{equation}
   \phi(x,\theta)
   =
   \frac{1}{\sqrt{2\pi}}
   \sum_{k}\tilde{\phi}_{k}(x)e^{ik\theta}
   \text{。}
\end{equation}
これを5次元の実スカラー場の作用\eqref{eqn:action_5d_real_scalar}に代入します。
\begin{equation}
   \int\dd^5 z\ 
   \sqrt{-g}
   =
   \int\dd^4 x\ \sqrt{-g_{4}}
   \cdot
   a\int_{0}^{2\pi}\dd \theta
\end{equation}
なので、$\theta$方向の積分を考えると
\begin{equation}
   \int_{0}^{2\pi}\dd \theta\ 
   \left(  
      -\frac{1}{2}g^{MN}(\partial_{M}\phi)(\partial_{N}\phi)
   \right)
   =
   \int_{0}^{2\pi}\dd \theta\ 
   \left(  
      -
      \frac{1}{2}\eta^{\mu\nu}(\partial_{\mu}\phi)(\partial_{\nu}\phi)
      -
      \frac{1}{2a^2}\partial_{\theta}\phi^2
   \right)   
   \label{eqn:2_1}
\end{equation}
となりますが、
\begin{equation}
   \partial_{\mu}\phi
   =
   \frac{1}{\sqrt{2\pi}}
   \sum_{k}\partial_{\mu}\tilde{\phi}_{k}(x)e^{ik\theta}
\end{equation}
より、
\begin{align}
   \int_{0}^{2\pi}\dd \theta\ 
   \left(  
      -
      \frac{1}{2}\eta^{\mu\nu}(\partial_{\mu}\phi)(\partial_{\nu}\phi)
   \right)   
   &=
   \int_{0}^{2\pi}\dd \theta\ 
   \left(  
      -\frac{1}{4\pi}\sum_{k,l}\partial_{\mu}\tilde{\phi}_{k}(x)\partial^{\mu}\tilde{\phi}_{l}(x)e^{i(k+l)\theta}
   \right)
   \nonumber
   \\
   &=
   -\frac{1}{2}\partial_{\mu}\tilde{\phi}_{0}(x)\partial^{\mu}\tilde{\phi}_{0}
   -\frac{1}{2}\sum_{k\neq 0}\partial_{\mu}\tilde{\phi}_{k}\partial^{\mu}\tilde{\phi}_{-k}
   \nonumber
   \\
   &=
   -\frac{1}{2}\partial_{\mu}\tilde{\phi}_{0}(x)\partial^{\mu}\tilde{\phi}_{0}
   -\sum_{k\geq 1}\partial_{\mu}\tilde{\phi}_{k}\partial^{\mu}\tilde{\phi}_{k}^{\ast}
\end{align}
です。ただし、$\phi(x,\theta)$が実スカラー場であることから
\begin{equation}
   \tilde{\phi}_{k}^{\ast}=\tilde{\phi}_{-k}
\end{equation}
を用いました。\eqref{eqn:2_1}の残りの項を考えると
\begin{align}
   \int_{0}^{2\pi}\dd \theta\ 
   \left(  
      -
      \frac{1}{2a^2}\partial_{\theta}\phi^2
   \right)   
   &=   
   \frac{1}{4\pi a^2}
   \int_{0}^{2\pi}\dd \theta\ 
   \sum_{k,l}kl\tilde{\phi}_{k}\tilde{\phi}_{l}e^{i(k+l)\theta}
   \nonumber
   \\
   &=
   -\sum_{k\leq 1}\left( \frac{|k|}{a} \right)^2 \tilde{\phi}_{k}\tilde{\phi}_{k}^{\ast}
\end{align}
となるので、4次元の有効作用は
\begin{equation}
   S
   =
   \int\dd^4 x\ a\mathcal{L}_{4}
\end{equation}
であり、ラグランジアンは
\begin{equation}
   \mathcal{L}_{4}
   =
   -\frac{1}{2}\partial_{\mu}\tilde{\phi}_{0}(x)\partial^{\mu}\tilde{\phi}_{0}
   -\sum_{n\geq 1}
   \left(  
      \partial_{\mu}\tilde{\phi}_{n}\partial^{\mu}\tilde{\phi}_{n}^{\ast}
      +
      M_{n}^{n}\tilde{\phi}_{n}\tilde{\phi}_{n}^{\ast}
   \right)
   \label{eqn:effective_Lagrangian_from_5d}
\end{equation}
となります。ただし、
\begin{equation}
   M_{n}
   \equiv
   \frac{|n|}{a}
\end{equation}
とおきました。

有効作用\eqref{eqn:effective_Lagrangian_from_5d}から次のことが分かります:
\begin{itemize}
   \item 
   5次元の実スカラー場の理論からは、質量$M_{n}$をもつ複素スカラー場(KK粒子)が現れること。
   \item 
   $n=0$のモードは、masslessであること(\textbf{ゼロモード})。
   \item 
   $n\neq 0$のときは、その粒子の質量は$M_{n}=|n|/a$であり、$a\ll 1$ならば$M_{n}\gg 1$であること。したがって、コンパクト空間の半径が非常に小さければ、十分なエネルギーがないとゼロモード以外の粒子を生成して観測することができないことになります\footnote{
      \href{https://pdg.lbl.gov/2023/tables/rpp2023-sum-searches.pdf}{Particle Date Groupのデータ}によると、TeVスケールではまだKK粒子の存在が確認されていないようです。
   }。
\end{itemize}

次は、(スピノル場の前に)ベクトル場$A_{M}(z)$の理論を見ていきます。ベクトル場の作用は、一番簡単なMaxwell理論
\begin{equation}
   S
   =
   \int\dd^5 z\ 
   \sqrt{-g}
   \left(  
      -\frac{1}{4}g^{MP}g^{NQ}F_{MN}F_{PQ}
   \right)
   \label{eqn:action_5d_maxwell}
\end{equation}
を考えます。ベクトル場は$z^{4}$の方向の周期境界条件により、スカラー場の場合と同様に
\begin{equation}
   A_{M}(x,\theta)
   =
   \frac{1}{\sqrt{2\pi}}
   \sum_{n}A_{M}^{(n)}(x)e^{in\theta}
\end{equation}
と展開できます。これを作用\eqref{eqn:action_5d_maxwell}に代入します。作用を少し書き換えると
\begin{align}
   S
   &=
   \int\dd^4 x\ \sqrt{-g_{4}}\ 
   \cdot
   a\int_{0}^{2\pi}\dd\theta\ 
   \left( -\frac{1}{4}g^{MP}g^{NQ}F_{MN}F_{PQ} \right)
   \nonumber
   \\
   &=
   \int\dd^4 x\ \sqrt{-g_{4}}\ 
   \cdot
   a\int_{0}^{2\pi}\dd\theta\ 
   \left(  
      -\frac{1}{4}(\partial^{\mu}A^{\nu}-\partial^{\nu}A^{\mu})^2
      -\frac{a}{2}(\partial^{\theta}A^{\mu}-\partial^{\mu}A^{\theta})^2
   \right)   
   \label{eqn:2_2}
\end{align}
となるので、被積分関数の各項は
\begin{align}
   (\partial^{\mu}A^{\nu}-\partial^{\nu}A^{\mu})^2
   &=
   \frac{1}{2\pi}
   \sum_{m,n}e^{i(m+n)\theta}
   \left(  
      \partial^{\mu}A^{\nu,(m)}
      -
      \partial^{\nu}A^{\mu,(m)}
   \right)
   \left(  
      \partial_{\nu}A_{\mu}^{(n)}
      -
      \partial_{\mu}A_{\nu}^{(n)}
   \right)
   \nonumber
   \\
   &\xrightarrow{\ \text{$\theta$方向の積分}\ }
   \sum_{n}
   \left(  
      \partial^{\mu}A^{\nu,(-n)}
      -
      \partial^{\nu}A^{\mu,(-n)}
   \right)
   \left(  
      \partial_{\nu}A_{\mu}^{(n)}
      -
      \partial_{\mu}A_{\nu}^{(n)}
   \right)   
   \\
   (\partial^{\theta}A^{\mu}-\partial^{\mu}A^{\theta})
   &=
   -
   \frac{1}{2\pi a^2}
   \sum_{m,n}e^{i(m+n)\theta}
   \left(  
      mA^{\mu,(m)}-\partial^{\mu}A^{\theta,(m)}
   \right)
   \left(  
      nA_{\mu}^{(n)}-\partial_{\mu}A_{\theta}^{(n)}
   \right)
   \nonumber
   \\
   &\xrightarrow{\ \text{$\theta$方向の積分}\ }
   \frac{1}{a^2}
   \sum_{n}
   \left(  
      nA^{\mu,(-n)}+\partial^{\mu}A^{\theta,(-n)}
   \right)
   \left(  
      nA_{\mu}^{(n)}-\partial_{\mu}A_{\theta}^{(n)}
   \right)
\end{align}
となります。ただし、
\begin{equation}
   \partial_{\theta}
   =
   \frac{1}{a}\pdv{}{\theta}
\end{equation}
として計算しています。この結果を、作用\eqref{eqn:2_2}の表式に代入すると
\begin{align}
   S
   &=
   \int\dd^4 x\ \sqrt{-g_{4}}\ 
   a
   \left(  
      -
      \frac{1}{4}
      \sum_{n}
      \left(  
         \partial^{\mu}A^{\nu,(-n)}
         -
         \partial^{\nu}A^{\mu,(-n)}
      \right)
      \left(  
         \partial_{\nu}A_{\mu}^{(n)}
         -
         \partial_{\mu}A_{\nu}^{(n)}
      \right)  
   \right.
   \nonumber
   \\
   &\hspace*{3.5cm}
   \left. 
      -
      \frac{1}{2a}   
      \sum_{n}
      \left(  
         nA^{\mu,(-n)}+\partial^{\mu}A^{\theta,(-n)}
      \right)
      \left(  
         nA_{\mu}^{(n)}-\partial_{\mu}A_{\theta}^{(n)}
      \right)   
   \right)
   \\
   &\equiv
   \int\dd^4 x\ \sqrt{-g_{4}}a\ \mathcal{L}_{4}
   \nonumber
\end{align}
です。このとき、ベクトル場の質量はどこからくるかというと
\begin{equation}
   \mathcal{L}_{4}
   \sim
   -\sum_{n}\frac{n^2}{2a}A^{\mu, (n)}A_{\mu}^{(n)\ast}
\end{equation}
の項からくるわけですが、やはり質量は$a^{-1}$に比例しています。したがって、effectiveなラグランジアンはゼロモードのみに注目すればよくて、それは
\begin{equation}
   \mathcal{L}_{\mathrm{eff}}
   =
   -\frac{1}{4}F^{\mu\nu}F_{\mu\nu}
   -\frac{1}{2a^2}\partial_{\mu}A_{\theta}\partial^{\mu}A_{\theta}^{\ast}
\end{equation}
です。ただし、$A_{M}^{(n=0)}$のモードの添え字は省略しており、$F_{\mu\nu}$は4次元の場の強さです。この表式から、4次元ではベクトル場のほかに複素スカラー場が生じていることが分かります\footnote{
   $A_{\theta}(x)$の添え字$\theta$は$4$次元のローレンツ変換では変換されません。したがって、この添え字は4次元の理論からは(ゲージ群の添え字と同様に)内部空間の添え字となるため、場$A_{\theta}(x)$はローレンツ変換に対してはスカラーです。
}。このように、より低いスピンをもつ粒子が生じるのがコンパクト化の特徴(だそう)です。




















% -----------------------

\clearpage
\makeatletter
\renewcommand{\appendix}{\par
  \setcounter{section}{0}%
  \setcounter{subsection}{0}%
  \gdef\presectionname{\appendixname}%
  \gdef\postsectionname{}%
  \gdef\thesection{\presectionname\@Alph\c@section\postsectionname}%
  \gdef\thesubsection{\@Alph\c@section.\@arabic\c@subsection}%
  \renewcommand{\theequation}{\@Alph\c@section.\arabic{equation}}%
  \renewcommand{\thefigure}{\@Alph\c@section.\arabic{figure}}%
  \renewcommand{\thetable}{\@Alph\c@section.\arabic{table}}%
}
\makeatother
\appendix


\section{標準模型の復習}

夜ゼミのネタになるかもしれないので、もしかしたらと思って書いておきます\footnote{
   夜ゼミがどんな感じかわかりませんが、真面目すぎたり簡単すぎたりしたらボツにします(笑)
}。

\subsection{非可換ゲージ理論}





\clearpage

\section{曲がった空間でのスピノル}












% -----------------------

\clearpage
\bibliography{ref}
\bibliographystyle{unsrt}

\nocite{Peskin:1995}
\nocite{Fujii:2005}

\end{document}
