\documentclass[unicode,a4paper,10pt]{ltjsarticle}
% ---fonts---
\PassOptionsToPackage{quiet}{fontspec}
\usepackage{luatexja-fontspec}
\setmainfont{TeX Gyre Termes}
\setmainjfont[BoldFont = HaranoAjiGothic-Regular]{HaranoAjiMincho}
% \setmainjfont[BoldFont = IPAGothic]{IPAMincho}
% \setmainjfont{Noto Sans CJK JP}
\setmathrm{Latin Modern Roman}

% ---Display \subsubsection at the Index
% \setcounter{tocdepth}{3}

% ---Setting about the geometry of the document----
% \usepackage{a4wide}
% \pagestyle{empty}

% ---Physics and Math Packages---
\usepackage{amssymb,amsfonts,amsthm,mathtools}
\usepackage{physics,braket,bm}

% ---underline---
\usepackage[normalem]{ulem}

% ---cancel---
\usepackage{cancel}

% --- surround the texts or equations
% \usepackage{fancybox,ascmac}

% ---settings of theorem environment---
\theoremstyle{definition}
\newtheorem{dfn}{定義}
\newtheorem{prop}{命題}
\newtheorem{thm}{定理}
\newtheorem{exm}{例}
\newtheorem{exc}{演習}

% ---settings of proof environment---
\renewcommand{\proofname}{\textbf{証明}}
\renewcommand{\qedsymbol}{$\blacksquare$}

% ---Ignore the Warnings---
\usepackage{silence}
\WarningFilter{latexfont}{Some font shapes}
\WarningFilter{latexfont}{Font shape}
\WarningFilter{latexfont}{Size substitutions}
\ExplSyntaxOn
\msg_redirect_name:nnn{hooks}{generic-deprecated}{none}
\ExplSyntaxOff

% ---Insert the figure (If insert the `draft' at the option, the process becomes faster.)---
\usepackage{graphicx}
% \usepackage{subcaption}

% ----Add a link to a text---
\usepackage{url,hyperref}
\usepackage[dvipsnames,svgnames]{xcolor}
\hypersetup{colorlinks=true,citecolor=FireBrick,linkcolor=Navy,urlcolor=purple}
% ---refer `texdoc xcolor' at the command line---

% ---Tikz---
\usepackage{tikz,pgf,pgfplots,circuitikz}
\pgfplotsset{compat=1.15}
\usetikzlibrary{intersections,arrows.meta,angles,calc,3d,decorations.pathmorphing}

% ---tcolorbox---
\usepackage{tcolorbox}
\tcbuselibrary{raster,skins,breakable}
\newtcolorbox{graybox}[1][]{frame empty, colback=black!07!white, sharp corners}

% ---Add the section number to the equation, figure, and table number---
\makeatletter
   \renewcommand{\theequation}{\thesection.\arabic{equation}}
   \@addtoreset{equation}{section}
   
   \renewcommand{\thefigure}{\thesection.\arabic{figure}}
   \@addtoreset{figure}{section}
   
   \renewcommand{\thetable}{\thesection.\arabic{table}}
   \@addtoreset{table}{section}
\makeatother

% ---enumerate---
% \renewcommand{\labelenumi}{$\arabic{enumi}.$}
% \renewcommand{\labelenumii}{$(\arabic{enumii})$}

% ---Index---
% \usepackage{makeidx}
% \makeindex 

\begin{document}

\maketitle

設問(1)\\

フリードマン方程式
\begin{equation}
  H^2
  =
  \frac{8\pi G}{3}\rho
  -
  \frac{Kc^2}{a^2}
  +
  \Lambda
\end{equation}
に$\rho=3M/4\pi a^3$を代入すれば
\begin{equation}
  \left(  
    \frac{\dot{a}}{a}
  \right)^2
  =
  \frac{2GM}{a^3}
  -
  \frac{Kc^2}{a^2}
  +
  \Lambda  
  \label{eqn:1-freedman}
\end{equation}
である。これを$Kc^2$について解けば
\begin{equation}
  Kc^2
  =
  a^2
  \left\{
    \frac{2GM}{a^3}
    -
    \left( \frac{\dot{a}}{a} \right)^2
    +
    \Lambda
  \right\}
  \label{eqn:1-1}
\end{equation}
となる。

フリードマン方程式\eqref{eqn:1-freedman}を微分すれば
\begin{equation}
  \frac{2a^2\dot{a}\ddot{a}-2a\dot{a}^3}{a^4}
  =
  -\frac{6GM}{a^4}+\frac{2Kc^2}{a^3}
\end{equation}
であり、これに\eqref{eqn:1-1}を代入して整理すると
\begin{equation}
  a^2\dot{a}\ddot{a}
  -
  a\dot{a}^3
  +
  a\dot{a}^2
  +
  GM
  -
  a^3\Lambda
  =
  0
\end{equation}
となる。\\


設問(2)\\

$a=a^{\ast}=\text{const.}$なら、$\dot{a}=0$なので、フリードマン方程式\eqref{eqn:1-1}と前問の結果は、それぞれ
\begin{gather}
  \frac{Kc^2}{{a^{\ast}}^2}
  =
  \frac{2GM}{{a^{\ast}}^3}
  +
  \Lambda
  \\
  GM
  -
  {a^{\ast}}^3\Lambda
  =
  0
\end{gather}
となり、$G$を消去して整理すれば
\begin{equation}
  {a^{\ast}}^2
  =
  \frac{Kc^2}{\Lambda}
  >
  0
\end{equation}
である。したがって、$K>0$であり、$a^{\ast}$は
\begin{equation}
  {a^{\ast}}
  =
  \sqrt{\frac{Kc^2}{\Lambda}}
\end{equation}
となる。\\


設問(3)\\

今は質点の質量$m$が$1$であるような状態を考えているので、$E$と$a$の関係は
\begin{equation}
  E
  =
  \frac{1}{2}\dot{a}^2
  -
  \frac{GM}{a}
\end{equation}
である。これに、フリードマン方程式\eqref{eqn:1-freedman}を代入すると
\begin{equation}
  
\end{equation}





\end{document}
