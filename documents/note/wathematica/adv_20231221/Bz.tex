\documentclass[a4paper,uplatex,dvipdfmx]{jsarticle}

% ---Display \subsubsection at the Index
% \setcounter{tocdepth}{3}

% ---Setting about the geometry of the document----
% \usepackage{a4wide}
% \pagestyle{empty}

% ---Physics and Math Packages---
\usepackage{amssymb,amsfonts,amsthm,mathtools}
\usepackage{physics,braket,bm}

% ---underline---
\usepackage{ulem}

% ---cancel---
\usepackage{cancel}

% --- surround the texts or equations
% \usepackage{fancybox,ascmac}

% ---settings of theorem environment---
\usepackage{amsthm}
\theoremstyle{definition}
\newtheorem{dfn}{定義}
\newtheorem{prop}{命題}
\newtheorem{thm}{定理}

% ---settings of proof environment---
\renewcommand{\proofname}{\textbf{証明}}
\renewcommand{\qedsymbol}{$\blacksquare$}

% ---Ignore the Warnings---
\usepackage{silence}
\WarningFilter{latexfont}{Some font shapes,Font shape}
\ExplSyntaxOn
\msg_redirect_name:nnn{hooks}{generic-deprecated}{none}
\ExplSyntaxOff

% ---Insert the figure (If insert the `draft' at the option, the process becomes faster.)---
\usepackage{graphicx}
% \usepackage{subcaption}

% ----Add a link to a text---
\usepackage{url,hyperref}
\usepackage[dvipsnames,svgnames]{xcolor}
\hypersetup{colorlinks=true,citecolor=FireBrick,linkcolor=Navy,urlcolor=purple}
\usepackage{pxjahyper}
% ---refer `texdoc xcolor' at the command line---

% ---Tikz---
\usepackage{tikz,pgf,pgfplots,circuitikz}
\pgfplotsset{compat=1.15}
\usetikzlibrary{intersections,arrows.meta,angles,calc,3d,decorations.pathmorphing}

% ---Add the section number to the equation, figure, and table number---
\makeatletter
   \renewcommand{\theequation}{\thesection.\arabic{equation}}
   \@addtoreset{equation}{section}
   
   \renewcommand{\thefigure}{\thesection.\arabic{figure}}
   \@addtoreset{figure}{section}
   
   \renewcommand{\thetable}{\thesection.\arabic{table}}
   \@addtoreset{table}{section}
\makeatother

% ---enumerate---
% \renewcommand{\labelenumi}{$\arabic{enumi}.$}
% \renewcommand{\labelenumii}{$(\arabic{enumii})$}

% ---Index---
% \usepackage{makeidx}
% \makeindex 

% ---Fonts---
% \renewcommand{\familydefault}{\sfdefault}
% \renewcommand{\kanjifamilydefault}{\gtdefault}

% ---Title---
\title{B'z}
\author{I. Miya}
\date{最終更新:\today}

\begin{document}

\section{\texorpdfstring{\includegraphics[width=0.023\linewidth]{fig/Bz/logo.png}}{B'z}%
のロゴの遷移}

本文の執筆中に「\includegraphics[width=0.020\linewidth]{fig/Bz/logo.png}のロゴ\footnote{
  文中のロゴ(\includegraphics[width=0.020\linewidth]{fig/Bz/logo.png}のこと)は\href{https://www.youtube.com/channel/UCpEEcJhBig2GqUFxcXjqiLg}{YouTubeのオフィシャルアカウント}からとってきた.
}\footnote{
  基本的にはシンプルなボールド体ではあるが,「z」の左上の部分が特徴的である.
}ってだいたいこれなのでは?(図\ref{STARS}, \ref{HighwayX})」と思ったので,このデザインがいつから使われているのかを調べてみた.なお,この章は本文とは独立しており,self-containedになるように心がけている.また,画像などのデータは\href{https://bz-vermillion.com/discography/}{ホームページのディスコグラフィ}を参考にしている.

\begin{figure}[ht]  
  \centering
  \begin{minipage}[t]{0.48\columnwidth}
    \centering
    \includegraphics[width=.7\hsize]{fig/Bz/stars_2023.png}
    \caption{シングル(2023年)}
    \label{STARS}
  \end{minipage}
  \begin{minipage}[t]{0.48\columnwidth}
    \centering
    \includegraphics[width=.7\hsize]{fig/Bz/highwayX_2022.png}
    \caption{アルバム(2022年)}
    \label{HighwayX}
  \end{minipage}
\end{figure}

ご存じの方は多いと思われるが,\includegraphics[width=0.02\linewidth]{fig/Bz/logo.png}は1988年に1stシングル『だからその手を離して』と1stアルバム『B'z』でデビューした\footnote{
  ちなみに,1988年は昭和63年であり,その翌年は平成元年(昭和64年)なので少し特殊な年である.こうやって覚えておくと,昭和と平成が変わった年がすぐ分かるので,年号に弱い方にはおすすめしたい.
}(図\ref{Bz}).1stシングル・アルバムのロゴは不思議なものが用いられており\ref{Bz},このイラストが用いられたのはこのときだけだろう.

\begin{figure}[ht]  
  \centering
  \begin{minipage}[t]{0.32\columnwidth}
    \centering
    \includegraphics[width=.7\hsize]{fig/Bz/bz.png}
    \caption{デビュー時のアルバム(肩パッドの時代)}
    \label{Bz}
  \end{minipage}
  \begin{minipage}[t]{0.32\columnwidth}
    \centering
    \includegraphics[width=.7\hsize]{fig/Bz/lovephantom.png}
    \caption{\includegraphics[width=0.05\linewidth]{fig/Bz/logo.png}が初めて使われたシングル(1995年)}
    \label{lovephantom}
  \end{minipage}
  \begin{minipage}[t]{0.32\columnwidth}
    \centering
    \includegraphics[width=.7\hsize]{fig/Bz/ichibutozenbu.png}
    \caption{2009年のシングル}
    \label{icibutozenbu}
  \end{minipage}
\end{figure}

本人たちも,自分たちのロゴを確定させたかったのだろうか,2nd以降はロゴのデザインが2~3回の周期で変化していった.










\end{document}
