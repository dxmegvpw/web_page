\documentclass[unicode,a4paper,10pt]{ltjsarticle}
% ---fonts---
\PassOptionsToPackage{quiet}{fontspec}
\usepackage{luatexja-fontspec}
\setmainfont{TeX Gyre Termes}
\setmainjfont[BoldFont = IPAGothic]{IPAMincho}
% \setmainjfont{Noto Sans CJK JP}
\setmathrm{Latin Modern Roman}

% ---Ignore the Warnings---
\usepackage{silence}
\WarningFilter{latexfont}{Some font shapes}
\WarningFilter{latexfont}{Font shape}
\WarningFilter{latexfont}{Size substitutions}
\ExplSyntaxOn
\msg_redirect_name:nnn{hooks}{generic-deprecated}{none}
\ExplSyntaxOff

\usepackage{graphics}
\usepackage[dvipsnames,svgnames]{xcolor}

\usepackage{tikz,pgf,pgfplots,circuitikz}
\pgfplotsset{compat=1.15}
\usetikzlibrary{intersections,arrows.meta,angles,calc,3d,decorations.pathmorphing}
\usepackage[compat=1.1.0]{tikz-feynhand}

\usepackage{amssymb,amsfonts,amsthm,mathtools}
\usepackage{physics,braket,bm}

% ---Title---
\title{Tikz サンプル}
\author{宮根一樹}
\date{\today}

\begin{document}

\maketitle
\tableofcontents

\clearpage
\section{はじめに}




\clearpage
\section{古典論としてのゲージ理論}

まずは、古典場の理論として、ゲージ理論がどのような役割を果たすのかについてみていきましょう。「電磁場はゲージ場のゲージ変換で不変」ということを電磁場で習ったことがあると思います。ある程度、場の理論に親しんだことのある方なら、この記述の意味が分かると思うのですが、初めてこれを聞いた人はそんなにピンとこないのではないかなと思います。私の記憶では、ゲージ不変であることから適切にゲージを選んで、実際にポテンシャルの満たす式が単純になるといった内容が電磁気学にあったと思います。もちろんこれらの内容は正しいですが、いくらか技巧的な文脈です。ゲージ理論は量子論へと応用することを考えると色々と面白い性質が見えてきます。

その足掛かりとして、この章では古典場としてのゲージ理論を見ていきましょう。

\subsection{古典場の理論の構成}




\clearpage

\subsection{可換ゲージ理論}


\clearpage

\subsection{非可換ゲージ理論}


\clearpage

\subsection{ゲージ理論的な一般相対性理論}










\clearpage
\section{ゲージ理論の量子化}









% ----------------------------------------
% \clearpage

% \makeatletter
% \renewcommand{\appendix}{\par
%   \setcounter{section}{0}%
%   \setcounter{subsection}{0}%
%   \gdef\presectionname{\appendixname}%
%   \gdef\postsectionname{}%
%   \gdef\thesection{\presectionname\@Alph\c@section\postsectionname}%
%   \gdef\thesubsection{\@Alph\c@section.\@arabic\c@subsection}%
%   \renewcommand{\theequation}{\@Alph\c@section.\arabic{equation}}%
%   \renewcommand{\thefigure}{\@Alph\c@section.\arabic{figure}}%
%   \renewcommand{\thetable}{\@Alph\c@section.\arabic{table}}%
% }
% \makeatother

% \appendix

% \section{Notes}


% ----------------------------------------
\clearpage
\bibliography{ref}
\bibliographystyle{ytphys}

\nocite{Peskin:1995}

% ----------------------------------------
% \clearpage
% \index{hoge@hoge}
% \printindex


\end{document}
