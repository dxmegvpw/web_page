\documentclass[unicode,a4paper,10pt]{ltjsarticle}
% ---fonts---
\PassOptionsToPackage{quiet}{fontspec}
\usepackage{luatexja-fontspec}
\setmainfont{TeX Gyre Termes}
\setmainjfont[BoldFont = HaranoAjiGothic-Regular]{HaranoAjiMincho}
% \setmainjfont[BoldFont = IPAGothic]{IPAMincho}
% \setmainjfont{Noto Sans CJK JP}
\setmathrm{Latin Modern Roman}

% ---Display \subsubsection at the Index
% \setcounter{tocdepth}{3}

% ---Setting about the geometry of the document----
% \usepackage{a4wide}
% \pagestyle{empty}

% ---Physics and Math Packages---
\usepackage{amssymb,amsfonts,amsthm,mathtools}
\usepackage{physics,braket,bm}

% ---underline---
\usepackage[normalem]{ulem}

% ---cancel---
\usepackage{cancel}

% --- surround the texts or equations
% \usepackage{fancybox,ascmac}

% ---settings of theorem environment---
\theoremstyle{definition}
\newtheorem{dfn}{定義}
\newtheorem{prop}{命題}
\newtheorem{thm}{定理}
\newtheorem{exm}{例}
\newtheorem{exc}{演習}

% ---settings of proof environment---
\renewcommand{\proofname}{\textbf{証明}}
\renewcommand{\qedsymbol}{$\blacksquare$}

% ---Ignore the Warnings---
\usepackage{silence}
\WarningFilter{latexfont}{Some font shapes}
\WarningFilter{latexfont}{Font shape}
\WarningFilter{latexfont}{Size substitutions}
\ExplSyntaxOn
\msg_redirect_name:nnn{hooks}{generic-deprecated}{none}
\ExplSyntaxOff

% ---Insert the figure (If insert the `draft' at the option, the process becomes faster.)---
\usepackage{graphicx}
% \usepackage{subcaption}

% ----Add a link to a text---
\usepackage{url,hyperref}
\usepackage[dvipsnames,svgnames]{xcolor}
\hypersetup{colorlinks=true,citecolor=FireBrick,linkcolor=Navy,urlcolor=purple}
% ---refer `texdoc xcolor' at the command line---

% ---Tikz---
\usepackage{tikz,pgf,pgfplots,circuitikz}
\pgfplotsset{compat=1.15}
\usetikzlibrary{intersections,arrows.meta,angles,calc,3d,decorations.pathmorphing}

% ---tcolorbox---
\usepackage{tcolorbox}
\tcbuselibrary{raster,skins,breakable}
\newtcolorbox{graybox}[1][]{frame empty, colback=black!07!white, sharp corners}

% ---Add the section number to the equation, figure, and table number---
\makeatletter
   \renewcommand{\theequation}{\thesection.\arabic{equation}}
   \@addtoreset{equation}{section}
   
   \renewcommand{\thefigure}{\thesection.\arabic{figure}}
   \@addtoreset{figure}{section}
   
   \renewcommand{\thetable}{\thesection.\arabic{table}}
   \@addtoreset{table}{section}
\makeatother

% ---enumerate---
% \renewcommand{\labelenumi}{$\arabic{enumi}.$}
% \renewcommand{\labelenumii}{$(\arabic{enumii})$}

% ---Index---
% \usepackage{makeidx}
% \makeindex 

\begin{document}

% ---Title---
\title{
  宇宙物理学
  \quad
  レポート2
}
\author{
  氏名:宮根一樹
  \quad
  学籍番号:5324A057-8
}
\date{最終更新:\today}

\maketitle

\begin{enumerate}
  \item 
  分子運動論を考える。まずは、$x$軸方向の運動のみに絞って考える。粒子が速度$v_{x}$で壁に弾性衝突したとすると、逆向きで速度$v_{x}$に運動を開始する。このときに粒子が受けた力を$F_{x}$とすると、運動量の変化と力積の関係から
  \begin{equation}
    F_{x}
    =
    2p_{x}\cdot\frac{v_{x}}{2}
  \end{equation}
  が成立する。このとき、単位時間あたりに$v_{x}/2$回だけ粒子が壁にぶつかることに注意する\footnote{
    立方体の各辺の長さも単位長さであることに注意する。
  }。これは$x$軸方向の関係のみなので、全ての方向を考えればそれは絶対値をとってから、その量の$1/3$を考えればよい。したがって、
  \begin{equation}
    \bar{P}
    =
    \frac{1}{3}pv
  \end{equation}
  である。

  \begin{figure}[ht]
    \centering
    \begin{minipage}{0.4\textwidth}
      \centering
      \includegraphics[width=0.8\textwidth]{fig/fig01.pdf}
      \caption{立方体の中での粒子の運動}      
    \end{minipage}
    \begin{minipage}{0.3\textwidth}
      \centering
      \includegraphics[width=0.8\textwidth]{fig/fig02.pdf}
      \caption{壁に衝突する瞬間}      
    \end{minipage}
  \end{figure}

  ここで、$\gamma\equiv1/\sqrt{1-(v/c)^2}$とおけば
  \begin{equation}
    p
    =
    \gamma mv
    \ ,\quad
    E
    =
    \gamma mc^2
  \end{equation}
  なので、$p=Ev/c^2$である。よって
  \begin{graybox}
    \begin{equation}
      \bar{P}
      =
      \frac{1}{3}\cdot\frac{Ev}{c^2}\cdot v
      =
      \frac{Ev^2}{3c^2}
    \end{equation}
  \end{graybox}
  である。また、$m=0$のときは、$E^2=m^2c^4+p^2c^2$から$E=cp$であり$v=c$。ゆえに
  \begin{graybox}
    \begin{equation}
      \bar{P}
      =
      \frac{1}{3}cp
    \end{equation}
  \end{graybox}
  である。


  \item 
  
  \item 
  以下の2つの方程式が与えられている:
  \begin{gather}
    \dot{\varepsilon}_{\textrm{DE}}
    +
    3H(1+w_{\textrm{DE}})\varepsilon_{\textrm{DE}}
    =
    0
    ,
    \label{eqn:continu}
    \\
    H^2
    =
    \frac{8\pi G}{3c^2}\varepsilon_{\textrm{DE}}
    .
    \label{eqn:freedman}
  \end{gather}
  \eqref{eqn:freedman}を微分すると
  \begin{equation}
    2H\dot{H}
    =
    \frac{8\pi G}{3c^2}\dot{\varepsilon}_{\textrm{DE}}
  \end{equation}
  となるので、\eqref{eqn:continu}を代入し、$\varepsilon_{\textrm{DE}}$に対して\eqref{eqn:freedman}を代入して整理すれば
  \begin{equation}
    \dv{}{t}H
    =
    -\frac{3}{2}(1+w_{\textrm{DE}})H^2
  \end{equation}
  となる。これを$H(t_{0})=H_{0}$のもとで解くと
  \begin{equation}
    \frac{1}{H(t)}
    -
    \frac{1}{H_{0}}
    =
    \frac{3}{2}(1+w_{\textrm{DE}})(t-t_{0})
  \end{equation}












\end{enumerate}

\end{document}
