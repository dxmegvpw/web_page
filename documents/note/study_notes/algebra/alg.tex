\pdfoutput=1
\documentclass[a4paper,pdftex,10pt]{report}
% \usepackage[whole,autotilde]{bxcjkjatype}
\usepackage[T1]{fontenc}
\usepackage{tgtermes}

% ---Display \subsubsection at the Index
\setcounter{tocdepth}{3}

% ---Setting about the geometry of the document----
\usepackage{a4wide}
% \pagestyle{empty}

% ---Physics and Math Packages---
\usepackage{amssymb,amsfonts,amsthm,mathtools}
\usepackage{physics,braket,bm,slashed}

% ---underline---
\usepackage[normalem]{ulem}

% ---cancel---
\usepackage{cancel}

% --- surround the texts or equations
\usepackage{fancybox,ascmac}

% ---settings of theorem environment---
\theoremstyle{definition}
\newtheorem{dfn}{Definition}
\newtheorem{prop}{Proposition}
\newtheorem{thm}{Theorem}

% ---settings of proof environment---
\renewcommand{\proofname}{\textbf{Proof}}
\renewcommand{\qedsymbol}{$\blacksquare$}

% ---Ignore the Warnings---
\usepackage{silence}
\WarningFilter{latexfont}{Some font shapes,Font shape}
\ExplSyntaxOn
\msg_redirect_name:nnn{hooks}{generic-deprecated}{none}
\ExplSyntaxOff

% ---Insert the figure (If insert the `draft' at the option, the process becomes faster.)---
% \usepackage{graphicx}
% \usepackage{subcaption}

% ----Add a link to a text---
\usepackage{url,hyperref}
\usepackage[dvipsnames,svgnames]{xcolor}
\hypersetup{colorlinks=true,citecolor=FireBrick,linkcolor=Navy,urlcolor=purple}

% ---Tikz---
\usepackage{tikz,pgf,pgfplots,circuitikz}
\pgfplotsset{compat=1.15}
\usetikzlibrary{intersections, arrows.meta, angles, calc, 3d, decorations.pathmorphing}
\usepackage[compat=1.1.0]{tikz-feynhand}

% ---tcolorbox---
\usepackage{tcolorbox}
\tcbuselibrary{raster,skins,breakable}
\newtcolorbox{graybox}[1][]{frame empty, colback=black!07!white, sharp corners}

% ---Add the section number to the equation, figure, and table number---
\makeatletter
   \renewcommand{\theequation}{\thesection.\arabic{equation}}
   \@addtoreset{equation}{section}
   
   \renewcommand{\thefigure}{\thesection.\arabic{figure}}
   \@addtoreset{figure}{section}
   
   \renewcommand{\thetable}{\thesection.\arabic{table}}
   \@addtoreset{table}{section}
\makeatother

% ---enumerate---
% \renewcommand{\labelenumi}{$\arabic{enumi}.$}
% \renewcommand{\labelenumii}{$(\arabic{enumii})$}

% ---Index---
% \usepackage{makeidx}
% \makeindex 

% ---footnotes---
\renewcommand{\thefootnote}{$\ast$\arabic{footnote}}

% ---Title---
\title{Notes on Analysis}
\author{Itsuki Miyane}
\date{Last modified:\ \today}

\begin{document}

\maketitle

\tableofcontents

\clearpage
\chapter{Representation theory}



\section{Lie group}

In subsequent sections, we will discuss the representation of the groups. Before going into such a subject, let us review the properties of groups, especially the Lie group.

\subsection{Definition of the Lie group}

One of the examples of the two-dimensional Lie group is the \textit{orthogonal group} $O(2,\mathbb{R})$. It can be written as
\begin{equation}
  O(2,\mathbb{R})
  =
  \left\{
  \left.
  \begin{pmatrix}
    \cos\theta & -\sin\theta \\
    \sin\theta & \cos\theta
  \end{pmatrix}
  ,\quad
  \begin{pmatrix}
    \cos\theta & -\sin\theta \\
    \sin\theta & \cos\theta
  \end{pmatrix}
  \begin{pmatrix}
    1 & 0  \\
    0 & -1
  \end{pmatrix}
  \right|
  0\leq\theta\leq 2\pi
  \right\}
  .
\end{equation}
The \textit{special orthogonal group} is constituted by elements of $O(2,\mathbb{R})$ whose determinant is unit:
\begin{equation}
  SO(2,\mathbb{R})
  =
  \left\{
  \left.
  \begin{pmatrix}
    \cos\theta & -\sin\theta \\
    \sin\theta & \cos\theta
  \end{pmatrix}
  \right|
  0\leq\theta\leq 2\pi
  \right\}
  .
\end{equation}
This group represents the rotation about the $x-y$ plain respect to the origin, and it also isomorphic to 
\begin{enumerate}
  \item
        the residual group $\mathbb{R}/\mathbb{Z}$, which is obtained by dividing the additional additive group $\mathbb{R}$ by its subgroup $\mathbb{Z}$,
  \item
        the group constituted by the complex number whose absolute value is $1$ for the product.
\end{enumerate}



















% ----------------------------------------
% \clearpage
% \appendix
% \section{Notes}

% ----------------------------------------
\clearpage
\bibliography{ref}
\bibliographystyle{ytamsalpha}

\nocite{Humphreys:1972}

% ----------------------------------------
% \clearpage
% \index{hoge@hoge}
% \printindex

\end{document}
