\documentclass[unicode,a4paper,10pt]{ltjsarticle}
% ---fonts---
\PassOptionsToPackage{quiet}{fontspec}
\usepackage{luatexja-fontspec}
\setmainfont{TeX Gyre Termes}
\setmainjfont[BoldFont = HaranoAjiGothic-Regular]{HaranoAjiMincho}
% \setmainjfont[BoldFont = IPAGothic]{IPAMincho}
% \setmainjfont{Noto Sans CJK JP}
\setmathrm{Latin Modern Roman}

% ---Display \subsubsection at the Index
% \setcounter{tocdepth}{3}

% ---Setting about the geometry of the document----
% \usepackage{a4wide}
% \pagestyle{empty}

% ---Physics and Math Packages---
\usepackage{amssymb,amsfonts,amsthm,mathtools}
\usepackage{physics,braket,bm}

% ---underline---
\usepackage[normalem]{ulem}

% ---cancel---
\usepackage{cancel}

% --- surround the texts or equations
% \usepackage{fancybox,ascmac}

% ---settings of theorem environment---
\theoremstyle{definition}
\newtheorem{dfn}{定義}
\newtheorem{prop}{命題}
\newtheorem{thm}{定理}
\newtheorem{exm}{例}
\newtheorem{exc}{演習}

% ---settings of proof environment---
\renewcommand{\proofname}{\textbf{証明}}
\renewcommand{\qedsymbol}{$\blacksquare$}

% ---Ignore the Warnings---
\usepackage{silence}
\WarningFilter{latexfont}{Some font shapes}
\WarningFilter{latexfont}{Font shape}
\WarningFilter{latexfont}{Size substitutions}
\ExplSyntaxOn
\msg_redirect_name:nnn{hooks}{generic-deprecated}{none}
\ExplSyntaxOff

% ---Insert the figure (If insert the `draft' at the option, the process becomes faster.)---
\usepackage{graphicx}
% \usepackage{subcaption}

% ----Add a link to a text---
\usepackage{url,hyperref}
\usepackage[dvipsnames,svgnames]{xcolor}
\hypersetup{colorlinks=true,citecolor=FireBrick,linkcolor=Navy,urlcolor=purple}
% ---refer `texdoc xcolor' at the command line---

% ---Tikz---
\usepackage{tikz,pgf,pgfplots,circuitikz}
\pgfplotsset{compat=1.15}
\usetikzlibrary{intersections,arrows.meta,angles,calc,3d,decorations.pathmorphing}

% ---tcolorbox---
\usepackage{tcolorbox}
\tcbuselibrary{raster,skins,breakable}
\newtcolorbox{graybox}[1][]{frame empty, colback=black!07!white, sharp corners}

% ---Add the section number to the equation, figure, and table number---
\makeatletter
   \renewcommand{\theequation}{\thesection.\arabic{equation}}
   \@addtoreset{equation}{section}
   
   \renewcommand{\thefigure}{\thesection.\arabic{figure}}
   \@addtoreset{figure}{section}
   
   \renewcommand{\thetable}{\thesection.\arabic{table}}
   \@addtoreset{table}{section}
\makeatother

% ---enumerate---
% \renewcommand{\labelenumi}{$\arabic{enumi}.$}
% \renewcommand{\labelenumii}{$(\arabic{enumii})$}

% ---Index---
% \usepackage{makeidx}
% \makeindex 

\begin{document}

% ---Title---
\title{
  アノマリーと指数定理の関係
}
\author{
  宮根 一樹
}
\date{最終更新:\today}

\maketitle
\tableofcontents

\section{はじめに}

これは、WathematicaでアノマリーとAtiyah-Singerの指数定理について紹介したときの参考資料です。

現在、高エネルギーの物質の運動を記述する枠組みとして、\textbf{場の量子論}というものが広く知られています。読んで字のごとく、この理論は場の理論を量子化するわけですが、このとき、量子化する前に存在していた対称性が、量子論では成立しないような状況というものが色々と調べられています。これらのものを\textbf{アノマリー}といいます。一般に、「アノマリーが存在していいのかどうか」については議論の余地があります。例えば、ゲージ対称性がアノマリーで破れてしまう場合には、そもそも知られている量子化の無矛盾性を破ってしまうことが知られているので、アノマリーが相殺されるような理論が要求されます。一方で、グローバル対称性の破れそのものは、量子論の無矛盾性には何も影響を及ぼさないことが多いです(少なくとも「グローバル対称性のアノマリーが相殺されるように」といった記述を見たことがありません)。また、アノマリーは摂動論から求められるダイナミクス自体にも影響を与えるため、現象論的にも重要になります。こういった事情から、アノマリーというのは場の量子論において色々と関心を集めてきました。

このアノマリーというものは、Adler\cite{Adler:1969gk}とBell-Jakiw\cite{Bell:1969ts}が、$\pi$中間子の光子への崩壊過程の散乱振幅を分析していたときに発見されたものです。このときは、ある散乱過程のファインマンダイアグラムを計算することでアノマリーを計算して見せたわけですが、かなり発見的な側面がありアノマリーに対しては十分な理解を与えてくれてはいませんでした。対称性と言えば、ラグランジアンの不変性を意味しているわけですから、その観点からアノマリーを計算してやることができればよいわけです。その定式化は、1979年に日本の物理学者が確立しており、現代では「藤川の方法」と言われています\cite{Fujikawa:1979ay}。これは、理論の分配関数に現れる経路積分の測度のヤコビアンとしてアノマリーが解釈できるというものでした。場の量子論は、経路積分と分配関数からすべての量を計算することができるため、この藤川先生の仕事により、色々な対称性のアノマリーが計算できるようになったわけです。

さて、長々と物理的な側面を記述してきたわけですが、この藤川の方法と指数定理には大きな関係があります。カイラルなゲージ理論では、ディラック作用素の指数がアノマリーに対応することが藤川の方法で判明します。そこで、このノートでは、
\begin{enumerate}
  \item 
  指数定理についての数学の知識を確認
  \item 
  藤川の方法を用いたカイラルアノマリーの計算
\end{enumerate}
を紹介したいと思います。参考文献としては、藤川先生が書いた本\cite{Fujikawa:2001b}と数学については\cite{Nakahara:2003}です。また、\cite{Peskin:1995,Nair:2005}も眺めました。


\section{ディラック作用素と指数定理}













% ----------------------------------------
\clearpage
\bibliography{ref}
\bibliographystyle{ytphys}

\end{document}
