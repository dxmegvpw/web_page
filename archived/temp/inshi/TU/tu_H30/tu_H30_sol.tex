\pdfoutput=1
\documentclass[a4paper,pdflatex,ja=standard]{bxjsarticle}

% ---Setting about the geometry of the document----
% \usepackage{a4wide}
% \pagestyle{empty}

% ---Physics and Math Packages---
\usepackage{amssymb,amsfonts,amsthm,mathtools}
\usepackage{physics,braket,bm}

% ---underline---
\usepackage{ulem}

% --- sorround the texts or equations
% \usepackage{fancybox,ascmac}

% ---settings of theorem environment---
% \usepackage{amsthm}
% \theoremstyle{definition}

% ---settings of proof environment---
% \renewcommand{\proofname}{\textbf{証明}}
% \renewcommand{\qedsymbol}{$\blacksquare$}

% ---Ignore the Warnings---
\usepackage{silence}
\WarningFilter{latexfont}{Some font shapes,Font shape}

% ---Insert the figure (If insert the `draft' at the option, the process becomes faster)---
\usepackage{graphicx}
% \usepackage{subcaption}

% ----Add a link to a text---
\usepackage{url}
\usepackage{xcolor,hyperref}
\hypersetup{colorlinks=true,citecolor=orange,linkcolor=blue,urlcolor=magenta}
\usepackage{bxcjkjatype}

% ---Tikz---
\usepackage{tikz,pgf,pgfplots,circuitikz}
\pgfplotsset{compat=1.15}
\usetikzlibrary{intersections,arrows.meta,angles,calc,3d,decorations.pathmorphing}

% ---Add the section number to the equation, figure, and table number---
\makeatletter
   \renewcommand{\theequation}{\thesubsection.\arabic{equation}}
   \@addtoreset{equation}{section}
   
   \renewcommand{\thefigure}{\thesection.\arabic{figure}}
   \@addtoreset{figure}{section}
   
   \renewcommand{\thetable}{\thesection.\arabic{table}}
   \@addtoreset{table}{section}
\makeatother

% ---enumerate---
\renewcommand{\labelenumi}{\arabic{enumi}.}
\renewcommand{\labelenumii}{(\roman{enumii})}

% ---Index---
% \usepackage{makeidx}
% \makeindex 

% ---Fonts---
\renewcommand{\familydefault}{\sfdefault}

% ---Title---
\title{東京大学\ 平成30年\ 物理学専攻\ 院試\ 解答例}
\author{ミヤネ}
\date{最終更新:\today}

\newcommand{\prb}[2]{
  \phantomsection
  \addcontentsline{toc}{subsection}{問題 #1: #2}
  \subsection*{第#1問\phantom{#2}}
  \setcounter{subsection}{#1}
  \setcounter{equation}{0}
}

\begin{document}

\maketitle

\tableofcontents

\clearpage

\section{数学パート}

\prb{1}{線形代数}
\begin{enumerate}
  \item 
  固有方程式を解けば$\lambda=0,a^2+b^2$.対応する固有ベクトルはそれぞれ
  \begin{equation}
    \begin{pmatrix}
      b \\
      -a
    \end{pmatrix}
    \ ,\ 
    \begin{pmatrix}
      b \\
      a
    \end{pmatrix}
    \ .
  \end{equation}

  \item 
  次のようにおけばうまくいく:
  \begin{equation}
    B
    =
    (\sqrt{a^2+b^2}\ 0)
    \ ,\ 
    V
    =
    \frac{1}{\sqrt{a^2+b^2}}
    \begin{pmatrix}
      a & b \\
      b & -a
    \end{pmatrix}
    \ .
  \end{equation}

  \item 
  $\tilde{B}$を
  \begin{equation}
    \tilde{B}
    =
    \begin{pmatrix}
      1/c \\
      0
    \end{pmatrix}
  \end{equation}
  とおけばOK.

  \item 
  $\tilde{A}\coloneqq V\tilde{B}$とおけば$A\tilde{A}=1$が成立.また
  \begin{equation}
    (\tilde{A}A)^{T}
    =
    (V\tilde{B}BV)^{T}
    =
    V(\tilde{B}B)^{T}V
    =
    V\tilde{B}BV
    =
    \tilde{A}A
  \end{equation}
  より,$\tilde{A}A$は実対称.

  \item 
  $X$の成分を$x_{1},\cdots,x_{n}$とおく.この問題は
  \begin{equation}
    f(x_{1},\cdots,x_{n})
    \coloneqq
    x_{1}^2+\cdots+x_{n}^2
    \ ,\ 
    g(x_{1},\cdots,x_{n})
    \coloneqq
    a_{1}x_{1}+\cdots+a_{n}x_{n}-v
  \end{equation}
  とおいたときに,拘束条件$g(x_{1},\cdots,x_{n})=0$のもとで$f(x_{1},\cdots,x_{n})$の最小値をもとめればよいので
  \begin{equation}
    L
    \coloneqq
    f(x_{1},\cdots,x_{n})
    -
    \lambda
    g(x_{1},\cdots,x_{n})
  \end{equation}
  とおいて,未定乗数法を用いる.$L$を微分すると
  \begin{equation}
    \left\{
    \begin{alignedat}{1}
      \pdv{L}{x_{i}}
      &=
      2x_{i}-\lambda a_{i}x_{i}
      \\
      \pdv{L}{\lambda}
      &=
      a_{1}x_{1}+\cdots+a_{n}x_{n}-v
    \end{alignedat}
    \right.
  \end{equation}
  なので,
  \begin{equation}
    x_{i}
    =
    \frac{a_{i}}{\|A\|}v
    \ ,\ 
    \lambda
    =
    \frac{2v}{\|A\|}
    \ ,\ 
    \|A\|
    \coloneqq
    a_{1}^{2}+\cdots+a_{n}^{2}
  \end{equation}
  ともとまるので,
  \begin{equation}
    X_{0}
    =
    \frac{v}{\|A\|}
    \begin{pmatrix}
      a_{1} \\
      \vdots \\
      a_{n}
    \end{pmatrix}
  \end{equation}
  である.つまり
  \begin{equation}
    \tilde{A}_{n}
    =
    \frac{1}{\|A\|}
    \begin{pmatrix}
      a_{1} \\
      \vdots \\
      a_{n}
    \end{pmatrix}
  \end{equation}
  であり,これは
  \begin{equation}
    A_{n}\tilde{A}_{n}
    =
    \frac{1}{\|A\|}\left\{  
      a_{1}\cdot a_{1}
      +
      \cdots
      +
      a_{n}\cdot a_{n}
    \right\}
    =
    1
  \end{equation}
  かつ
  \begin{equation}
    (\tilde{A}_{n}A_{n})_{ij}
    =
    \frac{a_{i}a_{j}}{\|A\|} 
    =
    (\tilde{A}_{n}A_{n})_{ji}
  \end{equation}
  を満たす.

  \item 
  固有値を$\lambda$として,対応する固有ベクトルを$v$とすれば$M^{T}Mv=\lambda v$が成立.左から$M$をかければ
  \begin{equation}
    MM^{T}(Mv)
    =
    \lambda (Mv)
  \end{equation}
  となり,$\lambda$は依然として固有値.ただし,固有ベクトルは変化するが.

  さて,$M=A_{n}^{T}A_{n}$ととってみよう.ここで,$\tilde{A}_{n}=A_{n}^{T}/\|A\|$であることに気をつければ
  \begin{equation}
    \left\{
      \begin{alignedat}{1}
        (A_{n}^{T}A_{n})^{T}A_{n}^{T}A_{n}
        &=
        A_{n}^{T}A_{n}\cdot\|A\| \tilde{A}_{n}A_{n}
        =
        \|A\|A_{n}^{T}A_{n}
        \\
        A_{n}^{T}A_{n}(A_{n}^{T}A_{n})^{T}
        &=
        (A_{n}^{T}A_{n})^2
      \end{alignedat}
    \right.
  \end{equation}
  の固有値は等しい.$A_{n}^{T}A_{n}$の固有値を$\lambda$とすれば,$(A_{n}^{T}A_{n})^2$の固有値は$\lambda^2$である.$\|A\|A_{n}^{T}A_{n}$の固有値は$\|A\|\lambda$なので\footnote{
    行列が$k$倍されると固有値も$k$倍される.固有ベクトルのノルムは関係ないですが,行列をスカラー倍するのは関係あるので注意です.
  },
  \begin{equation}
    \lambda^2=\|A\|\lambda
  \end{equation}
  が成立し,$A_{n}^{T}A_{n}$の固有値は$0,\|A\|$のみである\footnote{
    設問1.と一致しています.今回の解答は固有値の候補を述べているだけなので,本当はちゃんと存在も言わないといけないと思いますが,言及しなくてもよいでしょう.設問1.で$n=2$の場合は確認できていますし.
  }.

  \item   
  固有ベクトルは合計$n$個なので,対角化可能である.

\end{enumerate}

\subsection*{補足}
\begin{itemize}
  \item 
  設問4.をちゃんとチェックしてみると
  \begin{equation}
    \tilde{A}
    =
    V^{T}\tilde{B}
    =
    \frac{1}{a^{2}+b^{2}}
    \begin{pmatrix}
      a \\
      b
    \end{pmatrix}
  \end{equation}
  より,ちゃんと
  \begin{equation}
    \tilde{A}A
    =
    \frac{1}{a^{2}+b^{2}}
    \begin{pmatrix}
      a \\
      b
    \end{pmatrix}
    (a\ b)
    =
    \frac{1}{a^{2}+b^{2}}
    \begin{pmatrix}
      a^2 & ab \\
      ab & b^2
    \end{pmatrix}
  \end{equation}
  と対称行列になっています.

  \item 
  一般の場合をちゃんとチェックするのはきついですが,$n=3$くらいはやってみましょう.行列
  \begin{equation}
    A_{3}^{T}A_{3}
    =
    \begin{pmatrix}
      a_{1}^{2} & a_{1}a_{2} & a_{1}a_{3} \\
      a_{2}a_{1} & a_{2}^{2} & a_{2}a_{3} \\
      a_{3}a_{1} & a_{3}a_{2} & a_{3}^{2}
    \end{pmatrix}
  \end{equation}
  の固有方程式は
  \begin{align}
    &
    \begin{vmatrix}
      \lambda-a_{1}^{2} & -a_{1}a_{2} & -a_{1}a_{3} \\
      -a_{2}a_{1} & \lambda-a_{2}^{2} & -a_{2}a_{3} \\
      -a_{3}a_{1} & -a_{3}a_{2} & \lambda-a_{3}^{2}
    \end{vmatrix}
    \nonumber
    \\
    &\hspace{10pt}
    =
    (\lambda-a_{1}^{2})(\lambda-a_{2}^{2})(\lambda-a_{3}^{2})
    -
    2a_{1}^{2}a_{2}^{2}a_{3}^{2}
    -
    \left(  
      a_{1}^{2}a_{2}^{2}(\lambda-a_{3}^2)
      +
      a_{2}^{2}a_{3}^{2}(\lambda-a_{1}^2)
      +
      a_{3}^{2}a_{1}^{2}(\lambda-a_{2}^2)
    \right)
    \nonumber
    \\
    &\hspace{10pt}
    =
    \lambda^3
    -
    (a_{1}^2+a_{2}^2+a_{3}^2)\lambda^2
    =
    0
  \end{align}
  となるので
  \begin{equation}
    \lambda
    =
    0 \ ,\ a_{1}^2+a_{2}^2+a_{3}^2
  \end{equation}
  が固有値です.固有値$0$が重解となっています.これに対応する固有値は,
  \begin{align}
    \begin{pmatrix}
      -a_{1}^{2} & -a_{1}a_{2} & -a_{1}a_{3} \\
      -a_{2}a_{1} & -a_{2}^{2} & -a_{2}a_{3} \\
      -a_{3}a_{1} & -a_{3}a_{2} & -a_{3}^{2}
    \end{pmatrix}
    &\rightarrow
    \begin{pmatrix}
      a_{1} & a_{2} & a_{3} \\
      a_{1} & a_{2} & a_{3} \\
      a_{1} & a_{2} & a_{3}
    \end{pmatrix}
    \nonumber
    \\
    &\rightarrow
    \begin{pmatrix}
      a_{1} & a_{2} & a_{3} \\
      0 & 0 & 0 \\
      0 & 0 & 0 
    \end{pmatrix}
  \end{align}
  なので,$a_{1}x_{1}+a_{2}x_{2}+a_{3}x_{3}=0$を解けばよくて,これは2つのパラメター$s,t$を用いて
  \begin{equation}
    \begin{pmatrix}
      x_{1} \\
      x_{2} \\
      x_{3}
    \end{pmatrix}
    =
    s
    \begin{pmatrix}
      a_{2} \\
      -a_{1} \\
      0
    \end{pmatrix}
    +
    t
    \begin{pmatrix}
      a_{3} \\
      0 \\
      -a_{1}
    \end{pmatrix}
  \end{equation}
  と書けます\footnote{
    こういうときは,$x_{2}=0$とおいて$x_{1},x_{3}$をもとめ,今度は逆に$x_{3}=0$とすればOKです.
  }.よって,対応する固有ベクトルは
  \begin{equation}
    \begin{pmatrix}
      a_{2} \\
      -a_{1} \\
      0
    \end{pmatrix}
    \ ,\ 
    \begin{pmatrix}
      a_{3} \\
      0 \\
      -a_{1}
    \end{pmatrix}
  \end{equation}
  で2つです.

\end{itemize}

\clearpage
\prb{2}{偏微分方程式}
\begin{enumerate}
  \item 
  解の形を$u(t,x)=T(t)X(x)$と仮定すると
  \begin{equation}
    \frac{1}{X}\dv{X}{x}
    =
    -\frac{1}{T}\dv{T}{t}
  \end{equation}
  と変数分離できているので,$A,\lambda$を用いて
  \begin{equation}
    u(t,x)
    =
    Ae^{\lambda (x-t)}
  \end{equation}
  とかける.$t=0$で$u(0,x)=f(x)$だとすれば
  \begin{equation}
    u(t,x)
    =
    f(x)e^{-\lambda t}
  \end{equation}
  である.

  \item 
  \begin{enumerate}
    \item 
    $F$を
    \begin{equation}
      F=\frac{1}{2}u^2-\dv{u}{x}
    \end{equation}
    とおけばよい.

    \item 
    $F'=0$より
    \begin{equation}
      \frac{1}{2}u^2
      -
      \dv{u}{x}
      =
      C
    \end{equation}
    である.$x\rightarrow\infty$での境界条件より
    \begin{equation}
      \dv{u}{x}
      =
      \frac{1}{2}(u^2-W^2)
    \end{equation}
    なので\footnote{
      $x\rightarrow\infty$では
      \begin{equation}
        u(x)^2
        \rightarrow
        W^2
        \ ,\ 
        \dv{u}{x}
        \rightarrow
        0
      \end{equation}
      なので,$C=W^2/2$です.
    },
    \begin{equation}
      \left( \frac{1}{u-W}-\frac{1}{u+W} \right)
      \dv{u}{x}
      =
      W
    \end{equation}
    である.これを解けば
    \begin{equation}
      \left|\frac{u-W}{u+W}\right|
      =
      e^{Wx+D}
    \end{equation}
    である.ただし,$D$は定数.$x=0$での境界条件から$D=0$である.よって,もとめる解は
    \begin{equation}
      u(x)
      =
      \frac{1-e^{Wx}}{1+e^{Wx}}W
    \end{equation}
    である.

  \end{enumerate}

  \item 
  \begin{enumerate}
    \item 
    $u^{*}$を式(2)に代入すると
    \begin{align}
      \text{(左辺)}
      &=
      \pdv{u^{*}}{t}
      +
      u^{*}\pdv{u^{*}}{x}
      \nonumber
      \\
      &=
      -
      \pdv{s}{x}{t}
      +
      \pdv{s}{x}\pdv[2]{s}{x}
      \nonumber
      \\
      &=
      -\pdv{}{x}\left(  
        \frac{1}{2}\left( \pdv{s}{x} \right)^2
        +
        D\pdv[2]{s}{x}
      \right)
      +
      \pdv{s}{x}\pdv[2]{s}{x}
      \nonumber
      \\
      &=
      -D\pdv[3]{s}{x}
      =
      D\pdv[2]{u^{*}}{x}
      =
      \text{(右辺)}
    \end{align}
    となっている.

    \item 
    式(3)の両辺を計算してみると
    \begin{align}
      \text{(左辺)}
      &=
      \pdv{s^{*}}{t}
      =
      \pdv{S}{\phi}\pdv{\phi}{t}
      =
      D\pdv{S}{\phi}\pdv[2]{\phi}{x}
      \nonumber
      \\
      \text{(右辺)}
      &=
      \frac{1}{2}
      \left( \pdv{s^{*}}{x} \right)^2
      +
      D
      \pdv[2]{s^{*}}{x}
      \nonumber
      \\
      &=
      \frac{1}{2}
      \left( \pdv{S}{\phi} \right)^2
      \left( \pdv{\phi}{x} \right)^2
      +
      D\pdv{}{x}\left( \pdv{S}{\phi}\pdv{\phi}{x} \right)
      \nonumber
      \\
      &=
      \frac{1}{2}
      \left( \pdv{S}{\phi} \right)^2
      \left( \pdv{\phi}{x} \right)^2
      +
      D\left\{  
        \pdv{}{x}\left( \pdv{S}{\phi} \right)\pdv{\phi}{x}
        +
        \pdv{S}{\phi}\pdv[2]{\phi}{x}
      \right\}
      \nonumber
      \\
      &=
      D\pdv{S}{\phi}\pdv[2]{\phi}{x}
      +
      \left( \pdv{\phi}{x} \right)^2
      \left[  
        \frac{1}{2}\left( \pdv{S}{\phi} \right)^2
        +
        D\pdv[2]{S}{\phi}
      \right]
    \end{align}
    となっているので,
    \begin{equation}
      \frac{1}{2}\left( \pdv{S}{\phi} \right)^2
      +
      D\pdv[2]{S}{\phi}
      =
      0
    \end{equation}
    が$S(\phi)$が満たす微分方程式.$\phi$で積分すると
    \begin{equation}
      \pdv{S}{\phi}
      =
      \frac{2D}{\phi}
    \end{equation}
    となる\footnote{
      $\pdv*{S}{\phi}$を1つの関数とみなせば,これは変数分離できています.
    }.もう一回積分すれば
    \begin{equation}
      S(\phi)
      =
      2D\log |\phi|
    \end{equation}
    となる.

    \item 
    フーリエ変換すると
    \begin{equation}
      \frac{1}{\sqrt{2\pi}}
      \int_{-\infty}^{\infty}\left[  
        \pdv{\tilde{\phi}}{t}
        +
        Dk^2\tilde{\phi}
      \right]e^{ikx}
      \dd k
      =
      0
    \end{equation}
    となるので,$\tilde{\phi}(k,t)$が満たす微分方程式は
    \begin{equation}
      \pdv{\tilde{\phi}}{t}
      +
      Dk^2\tilde{\phi}
      =
      0
    \end{equation}
    である.これは
    \begin{equation}
      \frac{1}{\tilde{\phi}}
      \pdv{\tilde{\phi}}{t}
      =
      -Dk^2
    \end{equation}
    となり,これを解けば
    \begin{equation}
      \tilde{\phi}(k,t)
      =
      A(k)e^{-Dk^2 t}
    \end{equation}
    である.初期条件より$A(k)=g(k)$なので,
    \begin{equation}
      \tilde{\phi}(k,t)
      =
      g(k)e^{-Dk^2 t}
    \end{equation}
    がもとめる解である.

  \end{enumerate}
\end{enumerate}

\subsection*{補足}
\begin{itemize}
  \item 
  ここで解いたのは,1階の時間微分と2階の空間微分の微分方程式で,一番身近なのはシュレーディンガー方程式や熱伝導方程式でしょう.今回の解は
  \begin{equation}
    \tilde{\phi}(k,t)
    =
    g(k)e^{-Dk^2 t}
  \end{equation}
  で,これをフーリエ変換で戻すと
  \begin{equation}
    \phi(x,t)
    =
    \frac{1}{\sqrt{2\pi}}
    \int_{-\infty}^{\infty}
    g(k)e^{-Dk^2 t+ikx}
    \dd k
  \end{equation}
  となります.これ以上はちゃんと計算できませんが,例えば,$g(k)\equiv 1$の場合\footnote{
    つまり,$t=0$での$\phi(x,0)$が$\delta(x)$の場合に対応し,$x=0$に集中していた波の伝搬の様子がわかります.
  }は計算できて
  \begin{equation}
    \phi(x,t)
    =
    \frac{1}{\sqrt{2Dt}}e^{-\tfrac{x^2}{4Dt}}
  \end{equation}
  です.

\end{itemize}

\clearpage

\section{物理パート}

\prb{1}{量子力学}

\begin{enumerate}
  \item 
  行列表示は
  \begin{equation}
    H-E_{2}\bm{1}
    =
    \begin{pmatrix}
      E_{1}-E_{2} & V \\
      V & 0
    \end{pmatrix}
  \end{equation}
  である.よって,状態$\ket{2}$に対する期待値は
  \begin{equation}
    \mel**{2}{(H-E_2)}{2}
    =
    V
    ,\ 
    \mel**{2}{(H-E_2)^2}{2}
    =    
    V^2
    .
  \end{equation}

  \item 
  固有値は,
  \begin{equation}
    E_{\pm}
    =
    \frac{E_{1}+E_{2}}{2}
    \pm
    \sqrt{V^2+\left( \frac{E_{1}-E_{2}}{2} \right)^2}
  \end{equation}
  である.対応する固有ベクトルは
  \begin{equation}
    \ket{\psi_{\pm}}
    =
    \begin{pmatrix}
      2V \\
      -E_{1}+E_{2}
      \pm
      \sqrt{(E_{1}-E_{2})^2+4V^2}
    \end{pmatrix}
  \end{equation}
  なので,
  \begin{equation}
    \frac{\ev*{2|\psi_{\pm}}}{\ev*{1|\psi_{\pm}}}
    =
    \frac{E_{2}-E_{1}}{2V}
    \pm
    \sqrt{1+\left( \frac{E_{1}-E_{2}}{2V} \right)^2}
  \end{equation}
  である.

  \item 
  素直に計算してみれば
  \begin{align}
    \ev{\psi_{-}|\psi_{+}}
    &=
    \left(  
      2V \ 
      -E_{1}+E_{2}
      -
      \sqrt{(E_{1}-E_{2})^2+4V^2}
    \right)
    \begin{pmatrix}
      2V \\
      -E_{1}+E_{2}
      +
      \sqrt{(E_{1}-E_{2})^2+4V^2}
    \end{pmatrix}
    \nonumber
    \\
    &=
    4V^2
    +
    (E_{1}-E_{2})^2
    -
    ((E_{1}-E_{2})^2+4V^2)
    =
    0
  \end{align}
  なので,直交.

  \item 
  設問2.において,$E_{1}\rightarrow E_{2}-\mathcal{E}\lambda$を代入すれば
  \begin{equation}
    E_{\pm}(\lambda)
    =
    E_{2}-\frac{\mathcal{E}\lambda}{2}
    \pm
    \sqrt{V^2+\left( \frac{\mathcal{E}\lambda}{2} \right)^2}
  \end{equation}
  である.このとき,確かに
  \begin{equation}
    E_{+}(\lambda)
    -
    E_{-}(\lambda)
    =
    2\sqrt{V^2+\left( \frac{\mathcal{E}\lambda}{2} \right)^2}
    \geq
    2V
  \end{equation}
  である.

  \item 
  \begin{enumerate}
    \item 
    基底の変換公式は
    \begin{equation}
      \left\{
        \begin{alignedat}{1}
          \ket{\psi_{+}^{(\lambda)}}
          &=
          \cos(\theta(\lambda))
          \ket{1}
          +
          \sin(\theta(\lambda))
          \ket{2}
          \\
          \ket{\psi_{-}^{(\lambda)}}
          &=
          -\sin(\theta(\lambda))
          \ket{1}
          +
          \cos(\theta(\lambda))
          \ket{2}
        \end{alignedat}
      \right.
    \end{equation}
    である\footnote{
      $\ket{\psi_{+}^{(\lambda)}}$の形が決まっているので,直交性から$\ket{\psi_{-}^{(\lambda)}}$の形も決まってきます.問題は$\ket{\psi_{-}^{(\lambda)}}$の符号ですが,ひとまず$\lambda=-1$で,それぞれちょうど$\ket{1},\ket{2}$になるようにとっておきました.
    }.これを用いれば,
    \begin{equation}
      \ket{\psi(t)}
      =
      (c_{+}\cos(\theta(\lambda))-c_{-}\sin(\theta(\lambda)))
      \ket{1}
      +
      (c_{+}\sin(\theta(\lambda))+c_{-}\cos(\theta(\lambda)))
      \ket{2}
    \end{equation}
    となるので,
    \begin{equation}
      \begin{pmatrix}
        c_{1}(t) \\
        c_{2}(t)
      \end{pmatrix}
      =
      \begin{pmatrix}
        \cos(\theta(t/T)) & -\sin(\theta(t/T)) \\
        \sin(\theta(t/T)) & \cos(\theta(t/T))
      \end{pmatrix}
      \begin{pmatrix}
        c_{+}(t) \\
        c_{-}(t)
      \end{pmatrix}
    \end{equation}
    である.これをシュレーディンガー方程式$(5)$に代入すれば
    \begin{align}
      &i\hbar
      \left\{  
        -\frac{\theta^{\prime}(t/T)}{T}
        \begin{pmatrix}
          \sin(\theta(t/T)) & \cos(\theta(t/T)) \\
          -\cos(\theta(t/T)) & \sin(\theta(t/T))
        \end{pmatrix}
        \begin{pmatrix}
          c_{+}(t) \\
          c_{-}(t)
        \end{pmatrix}
      \right.
      \nonumber
      \\
      &\hspace{1.5cm}\left.
        +
        \begin{pmatrix}
          \cos(\theta(t/T)) & -\sin(\theta(t/T)) \\
          \sin(\theta(t/T)) & \cos(\theta(t/T))
        \end{pmatrix}
        \pdv{}{t}
        \begin{pmatrix}
          c_{+}(t) \\
          c_{-}(t)
        \end{pmatrix}
      \right\}
      \nonumber
      \\
      &\hspace{2cm}
      =
      \begin{pmatrix}
        E_{2}-\mathcal{E}t/T & V \\
        V & E_{2}
      \end{pmatrix}
      \begin{pmatrix}
        \cos(\theta(t/T)) & -\sin(\theta(t/T)) \\
        \sin(\theta(t/T)) & \cos(\theta(t/T))
      \end{pmatrix}
      \begin{pmatrix}
        c_{+}(t) \\
        c_{-}(t)
      \end{pmatrix}
      \label{shr1}
    \end{align}
    となる.

    \item 
    もちろん
    \begin{equation}
      c_{\pm}(t)
      =
      \tilde{c}_{\pm}(t)
      \exp\left[  
        -\frac{i}{\hbar}
        \int_{-T}^{t}
        \dd t^{\prime}
        E_{\pm}(t^{\prime}/T)
      \right]
    \end{equation}
    なので
    \begin{equation}
      \pdv{c_{\pm}}{t}
      =
      \left(  
        \pdv{\tilde{c}_{\pm}}{t}
        -
        \frac{i}{\hbar}E_{\pm}(t/T)\tilde{c}_{\pm}(t)
      \right)
      \exp\left[  
        -\frac{i}{\hbar T}
        \int_{-T}^{t}
        \dd t^{\prime}
        E_{\pm}(t^{\prime}/T)
      \right]
    \end{equation}
    と変換される.よって,\eqref{shr1}に代入すれば\begin{align}
      &i\hbar
      \left\{  
        -\frac{\theta^{\prime}(t/T)}{T}
        \begin{pmatrix}
          \sin(\theta(t/T)) & \cos(\theta(t/T)) \\
          -\cos(\theta(t/T)) & \sin(\theta(t/T))
        \end{pmatrix}
        \begin{pmatrix}
          \tilde{c}_{+}(t) \\
          \tilde{c}_{-}(t)
        \end{pmatrix}
      \right.
      \nonumber
      \\
      &\hspace{0.5cm}\left.
        +
        \begin{pmatrix}
          \cos(\theta(t/T)) & -\sin(\theta(t/T)) \\
          \sin(\theta(t/T)) & \cos(\theta(t/T))
        \end{pmatrix}
        \left(  
          \pdv{}{t}
          \begin{pmatrix}
            \tilde{c}_{+}(t) \\
            \tilde{c}_{-}(t)
          \end{pmatrix}
          -
          \frac{i}{\hbar}
          \begin{pmatrix}
            E_{+}(t/T) & 0 \\
            0 & E_{-}(t/T)
          \end{pmatrix}
          \begin{pmatrix}
            \tilde{c}_{+}(t) \\
            \tilde{c}_{-}(t)
          \end{pmatrix}
        \right)        
      \right\}
      \nonumber
      \\
      &\hspace{2cm}
      =
      \begin{pmatrix}
        E_{2}-\mathcal{E}t/T & V \\
        V & E_{2}
      \end{pmatrix}
      \begin{pmatrix}
        \cos(\theta(t/T)) & -\sin(\theta(t/T)) \\
        \sin(\theta(t/T)) & \cos(\theta(t/T))
      \end{pmatrix}
      \begin{pmatrix}
        \tilde{c}_{+}(t) \\
        \tilde{c}_{-}(t)
      \end{pmatrix}
      \label{shr2}
    \end{align}
    となる\footnote{
      $e^{\cdots}$は共通因子として全て落としています.
    }.ここで,$T\gg 1$なので,$1/T$の項を無視すれば
    \begin{align}
      &i\hbar\pdv{}{t}
      \begin{pmatrix}
        \tilde{c}_{+}(t) \\
        \tilde{c}_{-}(t)
      \end{pmatrix}      
      =
      \left\{  
        -
        \begin{pmatrix}
          E_{+}(t/T) & 0 \\
          0 & E_{-}(t/T)
        \end{pmatrix}
      \right.
      \nonumber
      \\
      &\quad
      \left.
        +
        \begin{pmatrix}
          E_{2}+V\sin 2\theta(t/T)
          -
          \mathcal{E}\lambda
          \cos^2\theta(t/T)
          &
          \frac{1}{2}\mathcal{E}
          \lambda
          \sin 2\theta(t/T)
          +
          V\cos 2\theta(t/T)
          \\
          \frac{1}{2}\mathcal{E}\lambda\sin 2\theta(t/T)
          +
          V\cos 2\theta(t/T)
          &
          E_{2}
          -
          \mathcal{E}\lambda\sin^2\theta(t/T)
          -
          V\sin2\theta(t/T)
        \end{pmatrix}
      \right\}
      \begin{pmatrix}
        \tilde{c}_{+}(t) \\
        \tilde{c}_{-}(t)
      \end{pmatrix}
    \end{align}
    である.ここで,$V\ll\mathcal{E}$より,$E_2-E_+\sim 0, E_2-E_-\sim \mathcal{E}t/T$
    \begin{equation}
      i\hbar\pdv{}{t}
      \begin{pmatrix}
        \tilde{c}_{+}(t) \\
        \tilde{c}_{-}(t)
      \end{pmatrix}      
      =
      \begin{pmatrix}
        V\sin2\theta - \dfrac{\mathcal{E} t}{T}\cos^2\theta
        &
        V\cos2\theta + \dfrac{\mathcal{E} t}{2T}\sin2\theta
        \vphantom{\frac{\frac{1}{2}}{2}}
        \\
        V\cos2\theta + \dfrac{\mathcal{E} t}{2T}\sin2\theta
        &
        -V\sin2\theta + \dfrac{\mathcal{E} t}{T}\cos^2\theta        
        \vphantom{\frac{\dfrac{1}{2}}{2}}
      \end{pmatrix}
      \begin{pmatrix}
        \tilde{c}_{+}(t) \\
        \tilde{c}_{-}(t)
      \end{pmatrix}
      \label{eqn}
    \end{equation}
    となる\footnote{
      かなりゴリゴリ計算しましたが,早い段階で$1/T$を落としてもよかったかもしれません.
    }.

  \end{enumerate}

  \item 

  \eqref{eqn}の両辺を$-T$から時刻$t$まで積分すれば
  \begin{align}
        \tilde{c}_{+}(t)
        &=        
        \tilde{c}_{+}(-T)
        +
        \int_{-T}^{t}
        \dd t'
        \frac{1}{i\hbar}
        \left\{  
          \left(  
            V\sin2\theta - \dfrac{\mathcal{E} t'}{T}\cos^2\theta
          \right)
          \tilde{c}_{+}(t')
          +
          \left( 
            V\cos2\theta + \dfrac{\mathcal{E} t'}{2T}\sin2\theta
          \right)
          \tilde{c}_{-}(t')
        \right\}
        \\
        \tilde{c}_{-}(t)
        &=
        \tilde{c}_{-}(-T)
        +
        \int_{-T}^{t}
        \dd t'
        \frac{1}{i\hbar}
        \left\{  
          \left(  
            V\cos2\theta + \dfrac{\mathcal{E} t'}{2T}\sin2\theta
          \right)
          \tilde{c}_{+}(t')
          +
          \left( 
            -V\sin2\theta + \dfrac{\mathcal{E} t'}{T}\cos^2\theta
          \right)
          \tilde{c}_{-}(t')
        \right\}
  \end{align}
  であるが,右辺は$t$が$-T$から離れると激しく振動する項と$\mathcal{O}(1)$で変化する項の積を積分したものとなっている\footnote{
    $t=0$付近で激しく振動するのは,問題文の図1から分かります.
  }.よって,積分の項の寄与はほとんどないとみなせるので,$|\tilde{c}_{+}(t)|\sim 1, |\tilde{c}_{-}(t)|\sim 0$である.

\end{enumerate}

\subsection*{補足}
\begin{itemize}
  \item 
  強引に$\ket{\psi_{+}^{\lambda}}$の係数を求めてみましょう.といっても,一般の場合の基底は分かっているので,代入してやるだけで
  \begin{equation}
    \ket{\psi_{\pm}^{(\lambda)}}
    =
    C_{\pm}
    \begin{pmatrix}
      2V \\
      \mathcal{E}\lambda
      \pm
      \sqrt{\mathcal{E}^2\lambda^2+4V^2}
    \end{pmatrix}
  \end{equation}
  ともとまります.ただし,$C_{\pm}$は規格化定数です.これを計算すると
  \begin{equation}
    C_{\pm}
    =
    \frac{1}{\sqrt{2(\mathcal{E}^2\lambda^2+4V^2\pm
    \mathcal{E}\lambda\sqrt{\mathcal{E}^2\lambda^2+4V^2})}}
  \end{equation}
  となるので,
  \begin{equation}
    \ket{\psi_{+}^{(\lambda)}}
    =
    \frac{1}{\sqrt{2(\mathcal{E}^2\lambda^2+4V^2+
    \mathcal{E}\lambda\sqrt{\mathcal{E}^2\lambda^2+4V^2})}}
    \begin{pmatrix}
      2V \\
      \mathcal{E}\lambda
      +
      \sqrt{\mathcal{E}^2\lambda^2+4V^2}
    \end{pmatrix}
  \end{equation}
  です.したがって,
  \begin{align}
    \sin(\theta(\lambda))
    &=
    \frac{2V}{\sqrt{2(\mathcal{E}^2\lambda^2+4V^2+
    \mathcal{E}\lambda\sqrt{\mathcal{E}^2\lambda^2+4V^2})}}
    \ ,
    \nonumber
    \\
    \cos(\theta(\lambda))
    &=
    \frac{\mathcal{E}\lambda
    +
    \sqrt{\mathcal{E}^2\lambda^2+4V^2}}{\sqrt{2(\mathcal{E}^2\lambda^2+4V^2+
    \mathcal{E}\lambda\sqrt{\mathcal{E}^2\lambda^2+4V^2})}}
  \end{align}
  となります.確かに,問題文中の図のようになりそうです.

\end{itemize}



\clearpage
\prb{2}{統計力学}

\begin{enumerate}
  \item 
  \begin{enumerate}
    \item 
    この系の分配関数は,古典的に
    \begin{equation}
      Z[\beta]
      =
      \frac{1}{h^{3N}N!}
      \int\dd^3 \bm{r}_{1}\cdots\dd^3 \bm{r}_{N}
      \int\dd^3 \bm{p}_{1}\cdots\dd^3 \bm{p}_{N}
      \exp\left[ -\beta\sum_{i=1}^{N}\frac{\bm{p}_{i}^{2}}{2m} \right]  
      =
      \frac{V^{N}}{h^{3N}N!}
      \left( \frac{2\pi m}{\beta} \right)^{3N/2}
    \end{equation}
    である.したがって,内部エネルギーと定積熱容量は
    \begin{equation}
      \left\{
        \begin{alignedat}{1}
          U
          &=
          -\pdv{}{\beta}
          \log Z[\beta]
          =
          \frac{3}{2}Nk_{B}T
          \\
          C_{V}
          &=
          \pdv{U}{T}
          =
          \frac{3}{2}Nk_{B}
        \end{alignedat}
      \right.
    \end{equation}
    である.

    \item 
    自由エネルギーは,
    \begin{equation}
      F
      =
      -k_{B}T\log Z
      =
      Nk_{B}T\log V
      +
      (\text{$V$に関係のない項})
    \end{equation}
    なので
    \begin{equation}
      p
      =
      -\pdv{F}{V}
      =
      \frac{Nk_{B}T}{V}
    \end{equation}
    である.

    \item 
    定積熱容量は温度に依存しないので
    \begin{equation}
      \Delta
      S(T_{0},T_{0}/2)
      =
      \frac{3}{2}Nk_{B}\cdot
      \int_{T_{0}/2}^{T_{0}}
      \dd T^{\prime}\ 
      \frac{1}{T^{\prime}}
      =
      \frac{3}{2}Nk_{B}\log 2
    \end{equation}
    と一定である.

  \end{enumerate}

  \item 
  \begin{enumerate}
    \item 
    大分配関数は
    \begin{equation}
      \Xi[\beta,\mu]
      =
      \sum_{N=0}^{\infty}
      \frac{1}{h^{3N}N!}
      \int\dd^3 \bm{r}_{1}\cdots\dd^3 \bm{r}_{N}
      \int\dd^3 \bm{p}_{1}\cdots\dd^3 \bm{p}_{N}
      \exp\left[ -\beta\left\{  
        \sum_{i=1}^{N}\frac{\bm{p}_{i}^{2}}{2m}
        -
        \mu N
      \right\} \right]
    \end{equation}
    である.この右辺を計算してやると
    \begin{equation}
      \Xi[\beta,\mu]
      =
      \exp\left[  
        \frac{(2\pi mk_{B}T)^{3/2}}{h^{3}V}e^{\mu/k_{B}T}
      \right]
    \end{equation}
    である.したがって,温度$T$のときの平均粒子数は
    \begin{equation}
      \ev{N(T)}
      =
      k_{B}T\pdv{}{\beta}
      \log\Xi[\mu,T]
      =
      \frac{(2\pi mk_{B}T)^{3/2}}{h^{3}V}e^{\mu/k_{B}T}
    \end{equation}
    なので,
    \begin{equation}
      \frac{\ev{N(T_{1})}}{\ev{N(T_{2})}}
      =
      \left( \frac{T_{1}}{T_{2}} \right)^{3/2}
      \exp\left[ 
        \frac{\mu}{k_{B}}\left( \frac{1}{T_{1}}-\frac{1}{T_{2}} \right) 
      \right]
    \end{equation}
    である.

    \item 
    圧力は
    \begin{equation}
      P
      =
      -
      \frac{1}{\beta}\pdv{}{V}
      \log\Xi
      =
      \frac{1}{\beta V}
      \cdot
      \frac{(2\pi mk_{B}T)^{3/2}}{h^{3}V}e^{\mu/k_{B}T}
      =
      \frac{\ev{N(T)}k_{B}T}{V}
    \end{equation}
    である.

  \end{enumerate}


  \item 

  \begin{enumerate}
    
    \item 

    もっともエネルギーが低いのは$(n_{x},n_{y},n_{z})=(1,1,1)$のときなので,基底状態のエネルギーは
    \begin{equation}
      E_0
      =
      E(1,1,1)
      =
      \frac{9\pi^2\hbar^2}{8mL^2}
    \end{equation}
    である.第1励起状態は
    \begin{equation}
      E_1
      =
      E(1,1,2)
      =
      \frac{3\pi^2\hbar^2}{2mL^2}
    \end{equation}
    である.
    
    \item 

    ボーズ粒子なので,1粒子のエネルギーは
    \begin{equation}
      U_1
      =
      E_0 D(E_0)f(E_0)
      +
      E_11 D(E_1)f(E_1)
      +
      \cdots
    \end{equation}
    である.ただし,$D(\varepsilon)$は状態密度\footnote{
      今回,状態密度は
      $$
        D(\varepsilon)
        =
        \frac{1}{4\pi}
        \left(  
          \frac{2mL^2}{\hbar^2}
        \right)^{3/2}
        \varepsilon^{1/2}
      $$
      です.いつもは球の体積ですが,今回は楕円体の体積であることに注意しましょう.
    }で$f(\varepsilon)$はボーズ分布.$k_BT\ll E_1-E_0$ならば,$E_1 f(E_1)\ll E_0 f_0$なので,基底状態でエネルギーの平均値を評価してもよい.よって,
    \begin{equation}
      U
      =
      N \times U_1
      \sim
      N E_0 D(E_0)f(E_0)
      =
      \frac{1}{4\pi}
      \left(  
        \frac{2mL^2}{\hbar^2}
      \right)^{3/2}
      \frac{NE_0^{3/2}}{e^{\beta E_0}-1}
    \end{equation}
    であり,$T$の依存性は古典論と異なる.

    \item 

    定積熱容量は
    \begin{equation}
      C
      =
      \pdv{U}{T}
      =
      \frac{N}{4\pi k_BT^2}
      \left(  
        \frac{2mL^2}{\hbar^2}
      \right)^{3/2}
      \frac{E_0^{5/2}e^{E_0/k_BT}}{(e^{E_0/k_BT}-1)^2}
    \end{equation}
    なので,$T=0$では$C=0$となってしまう.

    \item 
    
    公式を使えば
    \begin{equation}
      \Delta S(T_0, T_0/2)
      =
      \frac{N}{4\pi k_B}
      \left(  
        \frac{2mL^2}{\hbar^2}
      \right)^{3/2}
      \int_{T_0/2}^{T_0}
      \frac{1}{T^3}\cdot\frac{e^{E_0/k_BT}}{(e^{E_0/k_BT}-1)^2}
      \dd T
    \end{equation}
    である.ここで,$T_0\rightarrow 0$で,被積分関数は$0$に収束し積分区間も$0$になるので,$\Delta S\rightarrow 0$となる\footnote{
      ちゃんとやるなら,被積分関数を上から$\varepsilon$で抑えて積分して$\varepsilon\rightarrow 0$とすればいいでしょう.
    }.

    \item 

    古典論では$\Delta S=\mathrm{const.}$に対して,量子論では$T_0\rightarrow 0$で$\Delta S\rightarrow 0$となった.これは,古典論と量子論の状態数の考え方の違いからくる.ここでは,あるエネルギー$E$に対してとりうる状態の数を$\Omega(E)$と書くことにしよう.古典論では,状態数はphase spaceの表面積なので,$\Omega \propto E^{1/2} \propto T^{1/2}$である\footnote{
      各自由度に$k_B T/2$が分配されるので,$E\propto T$です.
    }.したがって,ボルツマンの関係式より
    $S\propto\log T$であり,$\Delta S_{\mathrm{classical}}=\mathrm{const.}$となる.一方,量子論では,状態数は状態空間の表面積であるが,十分低温では基底状態$(n_x,n_y,n_z)=(1,1,1)$のみが取りうる状態になる.したがって,$S=\mathrm{const.}$となり$\Delta S=0$となる\footnote{
      ちなみに,今回は箱の$z$軸方向の長さが$2L$なのはあまり効いてきませんでした.一応,第1励状態第の計算が影響を受けています.第1励起状態の縮退が変わったりするのですが,その影響が後半の議論に影響があるかというと$\cdots$,よくわかりません.
    }.    

  \end{enumerate}

\end{enumerate}


\clearpage
\prb{3}{電磁気学}

\begin{enumerate}
  \item 
  静電ポテンシャルは$\phi_{0}(\bm{r})=-\bm{p}\cdot\bm{r}=-qdr\cos\theta$.よって,
  \begin{equation}
    \bm{E}(\bm{r})
    =
    -\bm{\nabla}\phi_{0}(\bm{r})
    =
    qd(\cos\theta\bm{e}_{r}-\sin\theta\bm{e}_{\theta})
  \end{equation}
  である.

  \item 
  マクスウェル方程式を満たすようにとる:
  \begin{equation}
    \left\{
      \begin{alignedat}{1}
        \bm{E}
        &=
        -\bm{\nabla}\phi
        -
        \pdv{\bm{A}}{t}
        \\
        \bm{B}
        &=
        \bm{\nabla}\times\bm{A}
        \ .
      \end{alignedat}
    \right.
  \end{equation}

  \item 
  マクスウェル方程式(1)に,前問の第1式を代入すると
  \begin{equation}
    \nabla^2\phi
    -
    \frac{1}{c^2}\pdv[2]{\phi}{t}
    =
    -\frac{\rho}{\varepsilon_{0}}
  \end{equation}
  である.前問の第2式をマクスウェル方程式(4)に代入すれば
  \begin{equation}
    \bm{\nabla}(\bm{\nabla}\cdot\bm{A})
    -
    \nabla^2\bm{A}
    =
    \mu_{0}\bm{j}
    +
    \frac{1}{c^2}\pdv{}{t}
    \left(  
      -\bm{\nabla}\phi
      -
      \pdv{\bm{A}}{t}
    \right)
  \end{equation}
  なので,これを整理してローレンツの条件を代入すれば
  \begin{equation}
    \nabla^2
    \bm{A}
    -
    \frac{1}{c^2}\pdv[2]{\bm{A}}{t}
    =
    -\mu_{0}\bm{j}
  \end{equation}
  である.

  \item 
  $\bm{r}=(0,0,d/2)$における電流を調べれば
  \begin{equation}
    \dv{q}{t}
    =
    I_{0}e^{i\omega t}
  \end{equation}
  である.よって
  \begin{equation}
    q(t)
    =
    \frac{I_{0}}{i\omega}e^{i\omega t}
  \end{equation}
  である.

  \item 
  $\bm{j}(\bm{r},t)$は$z$成分しかもってないので,$\bm{A}$も$z$成分のみしか値をもたない.その値とは
  \begin{align*}
    A_{z}(\bm{r},t)
    &=
    \int
    \dd^3 \bm{r}^{\prime}\ 
    \frac{1}{|\bm{r}-\bm{r}^{\prime}|}
    \cdot
    I_{0}e^{i\omega (t-|\bm{r-r^{\prime}}|/c)}\delta(x^{\prime})\delta(y^{\prime
    })
    \nonumber
    \\
    &=
    \frac{\mu_{0}}{4\pi}
    \int\dd z^{\prime}\ 
    \frac{I_{0}e^{i\omega(t-\sqrt{x^2+y^2+(z-z^{\prime})^2/c})}}{\sqrt{x^2+y^2+(z-z^{\prime})^2}}
  \end{align*}
  である.ここで
  \begin{equation}
    \sqrt{x^2+y^2+(z-z^{\prime})^2}
    \sim
    r\left( 1-\frac{zz^{\prime}}{r^2} \right)
    =
    r
    -
    \frac{z}{r}z^{\prime}
  \end{equation}
  と近似すれば,積分は
  \begin{align}
    \int\dd z^{\prime}\quad
    \frac{I_{0}e^{i\omega(t-\sqrt{x^2+y^2+(z-z^{\prime})^2/c})}}{\sqrt{x^2+y^2+(z-z^{\prime})^2}}
    \sim
    \frac{I_{0}}{r}e^{i(\omega t-kr)}
    \int_{-d/2}^{+d/2}\dd z^{\prime}\ 
    e^{i(kz/r)z^{\prime}}
    =
    \frac{2I_{0}}{kz}e^{i(\omega t-kr)}\sin\left( \frac{dkz}{2r} \right)
  \end{align}
  となる
  \footnote{
    近似を用いれば,被積分関数の分母は
    $$
      \frac{1}{\sqrt{x^2+y^2+(z-z^{\prime})^2}}
      \sim
      \frac{1}{r}
      \cdot
      \left( 
        1
        +
        \frac{zz^{\prime}}{r^2}
        +
        \mathcal{O}(r^{-4})
       \right)
    $$
    と展開できますが,今回は「もっともゆっくり減衰する項」を取ってくればよいとのことなので,第1項の近似で打ち切りました.
  }.したがって,これを極座標に書き直せば
  \begin{equation}
    \bm{A}
    =
    A_{z}\bm{e}_{z}
    =
    \frac{\mu_{0}I_{0}}{2\pi kr}
    (\bm{e}_{r}-\tan\theta\bm{e}_{\theta})
    e^{i(\omega t-kr)}
    \sin\left( \frac{dk\cos\theta}{2} \right)
  \end{equation}
  である.よって,$\bm{B}(\bm{r},t)$は
  \begin{align}
    \bm{B}(\bm{r},t)
    &=
    \bm{\nabla}\times\bm{A}
    \nonumber
    \\
    &=
    \left( \bm{e}_{r}\pdv{}{r}+\bm{e}_{\theta}\frac{1}{r}\pdv{}{\theta}+\bm{e}_{\varphi}\frac{1}{r\sin\theta}\pdv{}{\varphi} \right)
    \times
    \left[  
      \frac{\mu_{0}I_{0}}{2\pi kr}
      (\bm{e}_{r}-\tan\theta\bm{e}_{\theta})
      e^{i(\omega t-kr)}
      \sin\left( \frac{dk\cos\theta}{2} \right)
    \right]
    \nonumber
    \\
    &=
    -\bm{e}_{\varphi}\tan\theta\frac{\mu_{0}I_{0}}{2\pi k}
    e^{i(\omega t-kr)}\sin\left( \frac{dk\cos\theta}{2} \right)
    \left( -\frac{i+ikr}{r^2} \right)
    \nonumber
    \\
    &\qquad
    -\bm{e}_{\varphi}\tan\theta
    \frac{\mu_{0}I_{0}}{2\pi k}e^{i(\omega t-kr)}\sin\left( \frac{dk\cos\theta}{2} \right)
    \nonumber
    \\
    &\sim
    \bm{e}_{\varphi}
    \frac{\mu_{0}I_{0}d\sin\theta}{4\pi r^2}
    e^{i(\omega t-kr)}
    \cos\left( \frac{dk\cos\theta}{2} \right)
  \end{align}
  である
  \footnote{
    次の関係式
    $$
      \bm{e}_{r}\times\bm{e}_{\theta}=\bm{e}_{\varphi}
      \quad ,\quad
      \bm{e}_{\varphi}\times\bm{e}_{r}=\bm{e}_{\theta}
      \quad ,\quad
      \bm{e}_{\theta}\times\bm{e}_{\varphi}=\bm{e}_{r}
    $$
    を用いる.これは,図を書くとよくわかるかも. また,$kr\gg 1$なので,第1項をおとした.
  }.また,$\bm{E}(\bm{r},t)$は
  \begin{align}
    \frac{1}{c^2}\pdv{\bm{E}}{t}
    &=
    \bm{\nabla}\times\bm{B}
    -
    \mu_{0}\bm{j}
    \nonumber
    \\
    &=
    \left( \bm{e}_{r}\pdv{}{r}+\bm{e}_{\theta}\frac{1}{r}\pdv{}{\theta}+\bm{e}_{\varphi}\frac{1}{r\sin\theta}\pdv{}{\varphi} \right)
    \times
    \bm{e}_{\varphi}
    \frac{\mu_{0}I_{0}d\sin\theta}{4\pi r^2}
    e^{i(\omega t-kr)}
    \cos\left( \frac{dk\cos\theta}{2} \right)
    \nonumber
    \\
    &\hspace*{3cm}
    +
    \mu_{0}j_{z}(\cos\theta\bm{e}_{r}-\sin\theta\bm{e}_{\theta})
    \nonumber
    \\
    &=
    -
    \bm{e}_{\theta}
    \frac{\mu_{0}I_{0}d\sin\theta}{4\pi}e^{i\omega t}\cos\left( \frac{dk\cos\theta}{2} \right)
    \pdv{}{r}\left( \frac{e^{-ikr}}{r} \right)
    \nonumber
    \\
    &\hspace*{3cm}
    +
    \bm{e}_{r}
    \frac{\mu_{0}I_{0}d}{4\pi r^3}
    e^{i(\omega t-kr)}
    \pdv{}{\theta}
    \left( \sin\theta\cos\left( \frac{dk\cos\theta}{2} \right) \right)
    \nonumber
    \\
    &\hspace*{3cm}
    +\mu_{0}I_{0}e^{i\omega t}\delta(x)\delta(y)(\cos\theta\bm{e}_{r}-\sin\theta\bm{e}_{\theta})
    \nonumber
    \\
    &\sim
    \bm{e}_{\theta}
    \frac{ik\mu_{0}I_{0}d\sin\theta}{4\pi r^2
    }e^{i(\omega t-kr)}\cos\left( \frac{dk\cos\theta}{2} \right)
    +\mu_{0}I_{0}e^{i\omega t}\delta(x)\delta(y)(\cos\theta\bm{e}_{r}-\sin\theta\bm{e}_{\theta})
  \end{align}
  より,
  \begin{equation}
    \bm{E}(\bm{r},t)
    =
    \bm{e}_{\theta}
    \frac{c\mu_{0}I_{0}d\sin\theta}{4\pi r^2
    }e^{i(\omega t-kr)}\cos\left( \frac{dk\cos\theta}{2} \right)
    +\frac{c^2\mu_{0}I_{0}}{i\omega}e^{i\omega t}\delta(x)\delta(y)(\cos\theta\bm{e}_{r}-\sin\theta\bm{e}_{\theta})
  \end{equation}
  である.

  \item 
  $\bm{E}=E_{r}\bm{e}_{r}+E_{\theta}\bm{e}_{\theta} , \bm{B}=B_{\varphi}\bm{e}_{\varphi}$とすると
  \begin{equation}
    \bm{S}
    =
    \frac{1}{\mu_{0}}
    \bm{E}\times\bm{B}
    =
    \frac{1}{\mu_{0}}\cdot
    B_{\varphi}
    \left[ -E_{r}\bm{e}_{\varphi}+E_{\theta}\bm{e}_{r} \right]
  \end{equation}
  である.$\bm{E},\bm{B}$の実部を調べると
  \begin{equation}
    \left\{
      \begin{alignedat}{1}
        E_{r}
        &=
        \frac{c^2\mu_{0}I_{0}}{\omega}\sin\omega t\cos\theta\delta(x)\delta(y)
        \\
        E_{\theta}
        &=
        \frac{c\mu_{0}I_{0}d\sin\theta}{4\pi r^2}
        \cos(\omega t-kr)\cos\left( \frac{dk\cos\theta}{2} \right)
        -
        \frac{c^2\mu_{0}I_{0}}{\omega}\sin\omega t\sin\theta\delta(x)\delta(y)
        \\
        B_{\varphi}
        &=
        \frac{\mu_{0}I_{0}d\sin\theta}{4\pi r^2}
        \cos(\omega t-kr)
        \cos\left( \frac{dk\cos\theta}{2} \right)
      \end{alignedat}
    \right.
  \end{equation}
  となっているので,ポインティングベクトルは
  \begin{align}
    \bm{S}(\bm{r},t)
    &=
    \frac{1}{\mu_{0}}\cdot
    \frac{\mu_{0}I_{0}d\sin\theta}{4\pi r^2}
    \cos(\omega t-kr)
    \cos\left( \frac{dk\cos\theta}{2} \right)
    \nonumber
    \\
    &\quad
    \times
    \left[  
      -
      \frac{c^2\mu_{0}I_{0}}{\omega}\sin\omega t\cos\theta\delta(x)\delta(y)
      \bm{e}_{\varphi}
    \right.
    \nonumber
    \\
    &\quad
    \left.
      +
      \left\{  
        \frac{c\mu_{0}I_{0}d\sin\theta}{4\pi r^2}
        \cos(\omega t-kr)\cos\left( \frac{dk\cos\theta}{2} \right)
        -
        \frac{c^2\mu_{0}I_{0}}{\omega}\sin\omega t\sin\theta\delta(x)\delta(y)
      \right\}
      \bm{e}_{r}
    \right]
  \end{align}
  である.これの時間平均をとれば,
  \begin{equation}
    \frac{\omega}{2\pi}\int_{0}^{2\pi/\omega}
    \dd t\ 
    \cos(\omega t-kr)\sin\omega t
    =
    \frac{\omega}{4\pi}\sin kr
    \int_{0}^{2\pi/\omega}
    \dd t\ 
    (1-\cos2\omega t)
    =
    \frac{1}{2}\sin kr
  \end{equation}
  より
  \begin{equation}
    \bar{\bm{S}}(\bm{r})
    =
    \frac{1}{2\pi/\omega}
    \int_{0}^{2\pi/\omega}
    \dd t
    \bm{S}(\bm{r},t)
    =
    -
    \frac{c^{2}\mu_{0}I_{0}^{2}d\sin\theta\sin kr}{8\pi\omega r^2}
    \cos\left( \frac{dk\cos\theta}{2} \right)
    \left[  
      \cos\theta
      \bm{e}_{\varphi}
      +
      \sin\theta
      \bm{e}_{r}
    \right]
    \delta(x)\delta(y)
  \end{equation}
  である.

  \item 
  向きは$\bm{e}_{z}$でそろっており,$E_{0}(R_{1})=E_{0}(R_{2})$なので
  \begin{equation}
    |\bm{E}|^{2}
    =
    |E_{0}(R_{1})|^2
    |\cos(\omega t-kR_{1}-\delta_{1})+\cos(\omega t-kR_{2}-\delta_{2})|^2
    \label{power}
  \end{equation}
  である.ここで
  \begin{equation}
    R_{1}
    \sim
    r+\frac{D}{2}\sin\varphi
    \ ,\ 
    R_{2}
    \sim
    r-\frac{D}{2}\sin\varphi
    \label{approx}
  \end{equation}
  と近似できるので,$\delta_{1}=\delta_{2}=0$なら
  \begin{equation}
    |\bm{E}|^2
    =
    4|E_{0}(R_{1})|^2
    \cos^2(\omega t-kr)
    \cos^2\left( \frac{kD}{2}\sin\varphi \right)
  \end{equation}
  である.よって,$\varphi=0$で最大で,最大値の半分になるためには
  \begin{equation}
    -\frac{\pi}{2kD}
    \leq
    \sin\varphi
    \leq
    \frac{\pi}{2kD}
  \end{equation}
  であればよい.この幅$\Delta\varphi$を減らすためには,$k$や$D$の値を大きくする,すなわち,波数を大きくしたり,波源の距離を大きくすればよい.また,一般の場合で(\ref{power})で近似(\ref{approx})を入れると
  \begin{equation}
    |\bm{E}|^2
    =
    |E_{0}(R_{1})|^2
    \left|
      \cos\left( \omega t-kr-\frac{kD}{2}\sin\varphi-\delta_{1} \right)
      +
      \cos\left( \omega t-kr+\frac{kD}{2}\sin\varphi-\delta_{2} \right)
    \right|^2
  \end{equation}
  となる.$\varphi=\varphi_{0}$で$|\cos(\cdots)+\cos(\cdots)|$の部分が最大となるためには,$\omega t-kr$以外の部分の値の差が$2\pi$の整数倍であればよい.つまり,$m$を整数とすれば
  \begin{equation}
    -\frac{kD}{2}\sin\varphi_{0}
    -
    \delta_{1}
    -
    \left( \frac{kD}{2}\sin\varphi_{0}
    -
    \delta_{2} \right)
    =
    2m\pi
  \end{equation}
  である.これを整理すれば
  \begin{equation}
    \delta_{2}
    -
    \delta_{1}
    =
    kD\sin\varphi_{0}
    +
    2m\pi
    \qquad
    (m\in\mathbb{Z})
  \end{equation}
  である.

\end{enumerate}


\end{document}
