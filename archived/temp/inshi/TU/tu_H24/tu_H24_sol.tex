\pdfoutput=1
\documentclass[a4paper,pdflatex,ja=standard]{bxjsarticle}

% ---Setting about the geometry of the document----
% \usepackage{a4wide}
% \pagestyle{empty}

% ---Physics and Math Packages---
\usepackage{amssymb,amsfonts,amsthm,mathtools}
\usepackage{physics,braket,bm}

% ---underline---
\usepackage{ulem}

% --- sorround the texts or equations
% \usepackage{fancybox,ascmac}

% ---settings of theorem environment---
% \usepackage{amsthm}
% \theoremstyle{definition}

% ---settings of proof environment---
% \renewcommand{\proofname}{\uline{\textbf{証明}}}
% \renewcommand{\qedsymbol}{$\blacksquare$}

% ---Ignore the Warnings---
\usepackage{silence}
\WarningFilter{latexfont}{Some font shapes,Font shape}

% ---Insert the figure (If insert the `draft' at the option, the process becomes faster)---
\usepackage{graphicx}
% \usepackage{subcaption}

% ----Add a link to a text---
\usepackage{url}
\usepackage{xcolor,hyperref}
\hypersetup{colorlinks=true,citecolor=orange,linkcolor=blue,urlcolor=magenta}
\usepackage{bxcjkjatype}

% ---Tikz---
\usepackage{tikz,pgf,pgfplots,circuitikz}
\pgfplotsset{compat=1.15}
\usetikzlibrary{intersections,arrows.meta,angles,calc,3d,decorations.pathmorphing}

% ---Add the section number to the equation, figure, and table number---
\makeatletter
   \renewcommand{\theequation}{\thesubsection.\arabic{equation}}
   \@addtoreset{equation}{subsection}
   
   \renewcommand{\thefigure}{\thesection.\arabic{figure}}
   \@addtoreset{figure}{section}
   
   \renewcommand{\thetable}{\thesection.\arabic{table}}
   \@addtoreset{table}{section}
\makeatother

% ---enumerate---
\renewcommand{\labelenumi}{\arabic{enumi}.}
\renewcommand{\labelenumii}{(\roman{enumii})}
\renewcommand{\labelenumiii}{(\alph{enumiii})}

% ---Index---
% \usepackage{makeidx}
% \makeindex

% ---Fonts---
\renewcommand{\familydefault}{\sfdefault}

% ---Title---
\title{東京大学\ 平成24年\ 物理学専攻\ 院試\ 解答例}
\author{ミヤネ}
\date{最終更新:\today}

\newcommand{\prb}[2]{
  \phantomsection
  \addcontentsline{toc}{subsection}{問題 #1: #2}
  \subsection*{第#1問\phantom{#2}}
  \setcounter{subsection}{#1}
  \setcounter{equation}{0}
}

\begin{document}

\maketitle

\tableofcontents
\clearpage


\section{数学パート}

\prb{1}{線形代数}

\begin{enumerate}

  \item 

  \begin{enumerate}

    \item 

    $\sigma_i,\sigma_j$の交換関係と反交換関係は
    \begin{equation}
      [\sigma_j,\sigma_k]
      =
      2i\varepsilon_{jki}\sigma_i
      ,\ 
      \{\sigma_j,\sigma_k\}
      =
      2\delta_{jk}
    \end{equation}
    なので,これらを足して2で割れば
    \begin{equation}
      \sigma_j\sigma_k
      =
      i\varepsilon_{jki}\sigma_i
      +
      \delta_{jk}
    \end{equation}
    となります.

    
    \item 

    \begin{align}
      S(\bm{a})S(\bm{b})
      &=
      a^{j}\sigma^{j}b^{k}\sigma^{k}
      \nonumber
      \\
      &=
      a^{j}b^{k}
      (
        i\varepsilon_{jki}\sigma_i
        +
        \delta_{jk}
      )
      \nonumber
      \\
      &=
      i\bm{\sigma}\cdot(\bm{a}\times\bm{b})
      +
      I\bm{a}\cdot\bm{b}
      =
      iS(\bm{a}\times\bm{b})
      +
      I\bm{a}\cdot\bm{b}
      .
    \end{align}


    \item 

    $S(\bm{n})^2=(\bm{n}\cdot\bm{\sigma})^2=1$なので,総和を偶数と奇数の項に分けると
    \begin{align}
      X(\bm{n},\theta)
      &=
      \sum_{k=0}^{\infty}
      \frac{(-i\theta)^k}{k!}S(\bm{n})^k
      \nonumber
      \\
      &=
      \sum_{k=0}^{\infty}
      \frac{(-i\theta)^{2k}}{(2k)!}
      +
      \sum_{k=0}^{\infty}
      \frac{(-i\theta)^{2k+1}}{(2k+1)!}S(\bm{n})
      \nonumber
      \\
      &=
      \sum_{k=0}^{\infty}
      \frac{(-1)^k\theta^{2k}}{(2k)!}
      -
      i
      \sum_{k=0}^{\infty}
      \frac{(-1)^k\theta^{2k+1}}{(2k+1)!}S(\bm{n})
      \nonumber
      \\
      &=
      I\cos \theta
      -
      i(\bm{n}\cdot\bm{\sigma})\sin\theta
    \end{align}
    となります.


    \item 

    \begin{align}
      &\quad
      X(\bm{n},\theta)S(\bm{v})X(\bm{n},-\theta)
      \nonumber
      \\
      &=
      (
        I\cos \theta
        -
        iS(\bm{n})\sin\theta
      )
      S(\bm{v})
      (
        I\cos \theta
        +
        iS(\bm{n})\sin\theta
      )
      \nonumber
      \\
      &=
      S(\bm{v})
      \cos^2\theta
      +
      i\sin\theta\cos\theta S(\bm{v})S(\bm{n})
      \nonumber
      \\
      &\hspace{2cm}
      -
      i\sin\theta\cos\theta S(\bm{n})S(\bm{v})
      +
      \sin^2\theta S(\bm{n})S(\bm{v})S(\bm{n})
      \nonumber
      \\
      &=
      S(\bm{v})
      \cos^2\theta
      +
      i\sin\theta\cos\theta
      \underbrace{
        (
          S(\bm{v})S(\bm{n})-S(\bm{n})S(\bm{v})
        )
      }_{=2iS(\bm{v}\times\bm{n})}
      +
      \sin^2\theta 
      \underbrace{
        \underbrace{
          S(\bm{n})S(\bm{v})S(\bm{n})
        }_{
          =
          S(\bm{n})(iS(\bm{v}\times\bm{n})+I\bm{v}\cdot\bm{n})
        }}_{
          =
          -
          \bm{\sigma}\cdot\bm{v}
          +
          (\bm{n}\cdot\bm{v})(\bm{\sigma}\cdot\bm{n})
          +
          S(\bm{n})(\bm{v}\cdot\bm{n})
      }
      \nonumber
      \\
      &=
      S(\bm{v})\cos^2\theta
      -
      \sin2\theta
      S(\bm{v}\times\bm{n})
      -
      \sin^2\theta
      \bm{\sigma}\cdot\bm{v}
      +
      \sin^2\theta
      (\bm{n}\cdot\bm{v})(\bm{\sigma}\cdot\bm{n})
      +
      \sin^2\theta
      S(\bm{n})(\bm{v}\cdot\bm{n})
      \nonumber
      \\
      &=
      S(\bm{v})\cos 2\theta
      -
      S(\bm{v}\times\bm{n})\sin 2\theta
      +
      2S(\bm{n})
      (\bm{v}\cdot\bm{n})
      \sin^2\theta
      \nonumber
      \\
      &=
      \sigma
      \cdot
      \underbrace{
        \left[  
          \bm{v}\cos 2\theta
          -
          (\bm{v}\times\bm{n})\sin 2\theta
          +
          2\bm{n}
          (\bm{v}\cdot\bm{n})
          \sin^2\theta
        \right]
      }_{\equiv \bm{v}^{\prime}}
      =
      S(\bm{v}^{\prime})
      .
    \end{align}


    \item 

    $\bm{n}\cdot\bm{v}=0$なので,
    \begin{equation}
      \bm{v}^{\prime}
      =
      \bm{v}\cos 2\theta
      -
      (\bm{v}\times\bm{n})\sin 2\theta
    \end{equation}
    です.これは,$\bm{v}$と$\bm{v}\times\bm{n}$の張る平面を,$\bm{v}^{\prime}$がクルクル回ることを意味しています.


  \end{enumerate}


  \item 

  \begin{enumerate}

    \item 

    真.$(AB)^{\dag}=B^{-1}A^{-1}=(AB)^{-1}$なので.

    \item 

    偽.$(AB)^{\dag}\neq AB$.

    \item 

    偽.
    $$
      A
      =
      \begin{pmatrix}
        1 & -i \\
        i & 2
      \end{pmatrix}
      .
    $$

    \item 

    偽.$A=A^{-1}$であれば良いので,$A=\sigma_{2}$とか.

    \item 

    偽.
    $$
      X
      =
      \begin{pmatrix}
        0 & 1 \\
        0 & 0
      \end{pmatrix}
      \neq 
      O
      .
    $$
    ($\det X=0$を探すとうまくいくかも?)

    \item 

    真.ケイリー・ハミルトンの定理より
    \begin{equation}
      X^2
      -
      (\tr X)X
      =
      O
      \label{KH}
    \end{equation}
    が成立します\footnote{
     行列$X$の固有多項式を$g_X$とおくと,
     $$
      g_X(\lambda)
      =
      \lambda^2
      -
      (\tr X) \lambda
      +
      \det X
     $$ 
     であり,$g_X(X)=0$が定理の内容です.また,今回は$X^3=0$の$\det$をとって$\det X=0$です.
    }.したがって,両辺に$X$を掛けると$(\tr X)X^2=0$.$\tr X=0$なら\eqref{KH}より$X^2=0$であり,$\tr X\neq 0$なら$X^2=0$


  \end{enumerate}


\end{enumerate}



\clearpage

\prb{2}{微分方程式}

\begin{enumerate}

  \item 

  \begin{enumerate}

    \item 

    行列$A$とその固有値は
    \begin{equation}
      A
      =
      \begin{pmatrix}
        0 & 4 \\
        1 & 0
      \end{pmatrix}
      ,\ 
      \lambda_1
      =
      2
      ,\ 
      \lambda_2
      =
      -2
    \end{equation}
    です.$\lambda_1,\ \lambda_2$に対応する固有ベクトルは
    \begin{equation}
      \bm{q}_1
      =
      \frac{1}{\sqrt{5}}
      \begin{pmatrix}
        2 \\
        1
      \end{pmatrix}
      ,\ 
      \bm{q}_2
      =
      \frac{1}{\sqrt{5}}
      \begin{pmatrix}
        -2 \\
        1
      \end{pmatrix}
    \end{equation}
    となります.


    \item 

    $P$に対して
    \begin{equation}
      P^{-1}AP
      =
      \begin{pmatrix}
        2 & 0 \\
        0 & -2
      \end{pmatrix}
    \end{equation}
    なので,偏微分方程式は
    \begin{equation}
      \pdv{}{t}
      \begin{pmatrix}
        s_1 \\
        s_2
      \end{pmatrix}
      =
      \begin{pmatrix}
        2 & 0 \\
        0 & -2
      \end{pmatrix}
      \pdv{}{x}
      \begin{pmatrix}
        s_1 \\
        s_2
      \end{pmatrix}
    \end{equation}
    です.


    \item 

    初期条件は
    \begin{equation}
      \begin{pmatrix}
        s_1 \\
        s_2
      \end{pmatrix}
      =
      P^{-1}
      \begin{pmatrix}
        u_1 \\
        u_2
      \end{pmatrix}
      =
      \frac{1}{2}
      \begin{pmatrix}
        1 & 2 \\
        -1 & 2
      \end{pmatrix}
      \begin{pmatrix}
        u_1 \\
        u_2
      \end{pmatrix}
    \end{equation}
    より,
    \begin{equation}
      s_1(x,0)
      =
      \frac{1}{2}e^{-x^2}
      ,\ 
      s_2(x,0)
      =
      -
      \frac{1}{2}e^{-x^2}
    \end{equation}
    です.偏微分方程式
    \begin{equation}
      \left\{
        \begin{alignedat}{1}
          \pdv{s_1}{t}
          +
          2
          \pdv{s_1}{x}
          &=
          0
          \\
          \pdv{s_2}{t}
          -
          2
          \pdv{s_2}{x}
          &=
          0
        \end{alignedat}
      \right.
    \end{equation}
    の解は
    \begin{equation}
      s_1(x,t)
      =
      f_1(x-2t)
      ,\ 
      s_1(x,t)
      =
      f_2(x+2t)
    \end{equation}
    という形で書けるので初期条件に気をつければ
    \begin{equation}
      s_1(x,t)
      =
      \frac{1}{2}e^{-(x-2t)^2}
      ,\ 
      s_2(x,t)
      =
      -
      \frac{1}{2}e^{-(x+2t)^2}
    \end{equation}
    と解けます.よって,
    \begin{equation}
      u_1(x,t)
      =
      \frac{1}{\sqrt{5}}
      \left(  
        e^{-(x-2t)^2}+e^{-(x+2t)^2}
      \right)
      ,\ 
      u_2(x,t)
      =
      \frac{2}{\sqrt{5}}
      \left(  
        e^{-(x-2t)^2}-e^{-(x+2t)^2}
      \right)
    \end{equation}
    が求める解です.また,$t=1$での様子は図\ref{time1}です.

    \begin{figure}[ht]
      \centering    
      \begin{tikzpicture}[scale=1.5] 
          \draw[->,>=stealth](0,-1.4)--(0,1.4)node[above]{$u(x,t)$};
          \draw[->,>=stealth](-2.8,0)--(2.8,0)node[below]{$x$};
          \draw[samples=100,domain=-2.8:2.8,ultra thin, red]plot(\x,{(1/sqrt(5))*(exp(-(\x-2)^2)+exp(-(\x+2)^2))})node[right]{$u_1(x,1)$};
          \draw[samples=100,domain=-2.8:2.8,ultra thin, blue]plot(\x,{(2/sqrt(5))*(exp(-(\x-2)^2)-exp(-(\x+2)^2))})node[right]{$u_2(x,1)$};
        \end{tikzpicture}    
        \caption{$t=1$での解の様子}
        \label{time1}
    \end{figure}

  \end{enumerate}

  \item 

  設問1と同様にやりましょう.$f=(f_1,f_2)$とすれば,方程式は
  \begin{equation}
    \pdv{}{t}
    \begin{pmatrix}
      f_1 \\
      f_2
    \end{pmatrix}
    +
    \begin{pmatrix}
      1 & -6 \\
      1 & -4
    \end{pmatrix}
    \pdv[2]{}{x}
    \begin{pmatrix}
      f_1 \\
      f_2
    \end{pmatrix}
    =
    0
  \end{equation}
  となるので,変換行列の固有値と固有ベクトルはそれぞれ
  \begin{equation}
    \lambda_1=-1
    ,\ 
    \bm{v}_1
    =
    \frac{1}{\sqrt{10}}
    \begin{pmatrix}
      3 \\
      1
    \end{pmatrix}
    ;\ 
    \lambda_2=-2
    ,\ 
    \bm{v}_2
    =
    \frac{1}{\sqrt{5}}
    \begin{pmatrix}
      2 \\
      1
    \end{pmatrix}
  \end{equation}
  です.よって,対角化する行列$Q$は
  \begin{equation}
    Q
    =
    \begin{pmatrix}
      3/\sqrt{10} & 2/\sqrt{5} \\
      1/\sqrt{10} & 1/\sqrt{5}
    \end{pmatrix}
  \end{equation}
  であり,関数を$r\equiv Q^{-1}f$と変換すれば,方程式は
  \begin{equation}
    \pdv{r_1}{t}-\pdv[2]{r_1}{x}=0
    ,\ 
    \pdv{r_2}{t}-2\pdv[2]{r_2}{x}=0
    \label{diff_eq}
  \end{equation}
  となります.ここで,$r_1,\ r_2$を
  \begin{equation}
    r_i(x,t)
    =
    \int\frac{\dd k}{2\pi}\tilde{r}_i(k,t)e^{-ikx}
    \quad
    (i=1,2)
    \label{fourier_transf}
  \end{equation}
  とフーリエ変換すれば,\eqref{diff_eq}は
  \begin{equation}
    \pdv{\tilde{r}_1}{t}
    +
    k^2\tilde{r}_1
    =
    0
    ,\ 
    \pdv{\tilde{r}_2}{t}
    +
    2
    k^2\tilde{r}_2
    =
    0
  \end{equation}
  となるので,これを解けば
  \begin{equation}
    \tilde{r}_1(k,t)
    =
    \tilde{r}_1(k,0)e^{-k^2 t}
    ,\ 
    \tilde{r}_2(k,t)
    =
    \tilde{r}_2(k,0)e^{-2k^2 t}
  \end{equation}
  が一般解となります.$r_1(x,t),\ r_2(x,t)$の初期条件は
  \begin{equation}
    \begin{pmatrix}
      r_1(x,0) \\
      r_2(x,0)
    \end{pmatrix}
    =
    \begin{pmatrix}
      \sqrt{10} & -2\sqrt{10} \\
      \sqrt{5} & 2\sqrt{5}
    \end{pmatrix}
    \begin{pmatrix}
      e^{-x^2} \\
      0
    \end{pmatrix}
  \end{equation}
  より,$r_1(x,0)=\sqrt{10}e^{-x^2},\ r_2(x,0)=\sqrt{5}e^{-x^2}$です.これを逆フーリエ変換すると
  \begin{align}
    \tilde{r}_1(k,0)
    &=
    \int\dd x\ 
    r_1(x,0) e^{ikx}
    =
    \sqrt{10\pi}e^{-k^2/4}
    \nonumber
    \\
    \tilde{r}_2(k,0)
    &=
    \int\dd x\ 
    r_2(x,0) e^{ikx}
    =
    \sqrt{5\pi}e^{-k^2/4}
  \end{align}
  となるので,フーリエモードは
  \begin{equation}
    \tilde{r}_1(k,t)
    =
    \sqrt{10\pi}\exp
    \left[  
      -\left( t+\frac{1}{4} \right)k^2
    \right]
    ,\ 
    \tilde{r}_2(k,t)
    =
    \sqrt{5\pi}\exp
    \left[  
      -2\left( t+\frac{1}{8} \right)k^2
    \right]
  \end{equation}
  となります.よって,関数$r_1,\ r_2$の解は
  \begin{align}
    r_1(x,t)
    &=
    \int\frac{\dd k}{2\pi}
    \sqrt{10\pi}\exp
    \left[  
      -\left( t+\frac{1}{4} \right)k^2
      -ikx
    \right]    
    \nonumber
    \\
    &=
    \sqrt{\frac{10}{4t+1}}\exp\left[ -\frac{x^2}{4t+1} \right]
    \\
    r_2(x,t)
    &=
    \int\frac{\dd k}{2\pi}
    \sqrt{5\pi}\exp
    \left[  
      -2\left( t+\frac{1}{8} \right)k^2
      -ikx
    \right]
    \nonumber
    \\
    &=
    \sqrt{\frac{10}{8t+1}}\exp\left[ -\frac{x^2}{8t+1} \right]
  \end{align}
  となります.最後に,$f=Qr$と変換すれば
  \begin{equation}
    \left\{
      \begin{alignedat}{1}
        f_1(x,t)
        &=
        \frac{3}{\sqrt{4t+1}}\exp\left[ -\frac{x^2}{4t+1} \right]
        +
        \sqrt{\frac{2}{8t+1}}\exp\left[ -\frac{x^2}{8t+1} \right]
        \\
        f_2(x,t)
        &=
        \frac{1}{\sqrt{4t+1}}\exp\left[ -\frac{x^2}{4t+1} \right]
        +
        \frac{1}{\sqrt{2(8t+1)}}\exp\left[ -\frac{x^2}{8t+1} \right]
      \end{alignedat}
    \right.
  \end{equation}
  と求める解が得られます.


\end{enumerate}



\clearpage


\section{物理パート}

\prb{1}{量子力学}

\begin{enumerate}

  \item 

  $L_i=i\hbar\varepsilon_{ijk}x_jp_k$なので,
  \begin{align}
    [L_i,L_j]
    &=
    \varepsilon_{ikl}\varepsilon_{jmn}
    [x_kp_l,x_mp_n]
    \nonumber
    \\
    &=
    i\hbar
    (
      \varepsilon_{ikl}\varepsilon_{jmk}x_mp_l
      +
      \varepsilon_{ikl}\varepsilon_{jlm}x_kn
    )
    \nonumber
    \\
    &=
    i\hbar\varepsilon_{ijk}\varepsilon_{klm}x_lp_m
    =
    i\hbar \varepsilon_{ijk}L_k
  \end{align}
  です.


  \item 

  前問より
  \begin{align}
    S_y
    &=
    \frac{i\hbar}{4}
    \begin{pmatrix}
      0 & 1 \\
      1 & 0
    \end{pmatrix}
    \begin{pmatrix}
      1 & 0 \\
      0 & -1
    \end{pmatrix}
    -
    \frac{i\hbar}{4}
    \begin{pmatrix}
      1 & 0 \\
      0 & -1
    \end{pmatrix}
    \begin{pmatrix}
      0 & 1 \\
      1 & 0
    \end{pmatrix}
    \nonumber
    \\
    &=
    \frac{\hbar}{2}
    \begin{pmatrix}
      0 & -i \\
      i & 0
    \end{pmatrix}
  \end{align}
  です.


  \item 

  ハミルトニアンは
  \begin{equation}
    \mathcal{H}
    =
    -\frac{\mu H}{2}
    \begin{pmatrix}
      0 & 1 \\
      1 & 0
    \end{pmatrix}
  \end{equation}
  となるので,固有値は
  \begin{equation}
    \lambda_1
    =
    -\frac{\mu H}{2}
    ,\ 
    \lambda_2
    =
    +\frac{\mu H}{2}
  \end{equation}
  であり,対応する固有ベクトルは
  \begin{equation}
    \ket{\lambda_1}
    =
    \frac{1}{\sqrt{2}}
    \begin{pmatrix}
      1 \\
      1
    \end{pmatrix}
    ,\ 
    \ket{\lambda_2}
    =
    \frac{1}{\sqrt{2}}
    \begin{pmatrix}
      1 \\
      -1
    \end{pmatrix}
  \end{equation}
  となります.


  \item 

  $\ket{\phi(0)}$を$\ket{\lambda_1},\ \ket{\lambda_2}$で展開すると
  \begin{equation}
    \ket{\phi(0)}
    =
    \frac{1}{\sqrt{2}}
    \ket{\lambda_1}
    +
    \frac{1}{\sqrt{2}}
    \ket{\lambda_2}
  \end{equation}
  となるので,
  \begin{align}
    \ket{\phi(t)}
    &=
    \frac{1}{\sqrt{2}}
    e^{-i\mathcal{H}t}
    \ket{\lambda_1}
    +
    \frac{1}{\sqrt{2}}
    e^{-i\mathcal{H}t}
    \ket{\lambda_2}
    \nonumber
    \\
    &=
    \frac{1}{\sqrt{2}}
    e^{+i\mu Ht/2}
    \ket{\lambda_1}
    +
    \frac{1}{\sqrt{2}}
    e^{-i\mu Ht/2}
    \ket{\lambda_2}
    \nonumber
    \\
    &=
    \begin{pmatrix}
      \cos \omega_0 t \\
      i\sin \omega_0 t 
    \end{pmatrix}
  \end{align}
  となります.ただし,
  \begin{equation}
    \omega_0
    \equiv
    \frac{\mu H}{2}
  \end{equation}
  とおきました.


  \item 

  $S_x$は簡単で
  \begin{align}
    \mel*{\phi(t)}{S_x}{\phi(t)}
    &=
    \frac{1}{\sqrt{2}}\mel*{\lambda_1}{S_x}{\lambda_1}
    +
    \frac{1}{\sqrt{2}}\mel*{\lambda_2}{S_x}{\lambda_2}
    \nonumber
    \\
    &=
    0
  \end{align}
  となります.残り$S_y,\ S_z$の期待値は,$\ket{\phi(t)}$の表示を用いて
  \begin{align}
    \mel*{\phi(t)}{S_y}{\phi(t)}    
    &=
    \frac{1}{2}
    \begin{pmatrix}
      \cos\omega_0t & -i\sin\omega_0t
    \end{pmatrix}
    \begin{pmatrix}
      0 & -i \\
      i & 0
    \end{pmatrix}
    \begin{pmatrix}
      \cos\omega_0t \\
      i\sin\omega_0t
    \end{pmatrix}
    =
    \frac{1}{2}\sin 2\omega_0t
    \\
    \mel*{\phi(t)}{S_z}{\phi(t)}    
    &=
    \frac{1}{2}
    \begin{pmatrix}
      \cos\omega_0t & -i\sin\omega_0t
    \end{pmatrix}
    \begin{pmatrix}
      1 & 0 \\
      0 & -1
    \end{pmatrix}
    \begin{pmatrix}
      \cos\omega_0t \\
      i\sin\omega_0t
    \end{pmatrix}
    =
    \frac{1}{2}\cos 2\omega_0t
  \end{align}
  となります.


  \item 

  シュレーディンガー方程式は
  \begin{equation}
    i
    \pdv{}{t}
    \ket{\varphi(t)}
    =
    (\mathcal{H}_1+\mathcal{H}_2)
    \ket{\varphi(t)}
  \end{equation}
  ですが,これに$\ket{\varphi(t)}=e^{-i\mathcal{H}_1t}\ket{\psi(t)}$を代入すれば
  \begin{equation}
    i\pdv{}{t}\ket{\psi(t)}
    =
    e^{+i\mathcal{H}_1t}
    \mathcal{H}_2
    e^{-i\mathcal{H}_1t}
    \ket{\psi(t)}
  \end{equation}
  が求める時間発展の方程式です.


  \item 

  $e^{+i\mathcal{H}_1t},\ \mathcal{H}_2$を計算すると
  \begin{align}
    e^{+i\mathcal{H}_1t}
    &=
    \sum_{k=0}^{\infty}
    \frac{(-iat/2)^k}{k!}
    \begin{pmatrix}
      1 & 0 \\
      0 & -1
    \end{pmatrix}^k
    \nonumber
    \\
    &=
    \sum_{k=0}^{\infty}
    \frac{(-1)^k(at/2)^{2k}}{(2k)!}
    \begin{pmatrix}
      1 & 0 \\
      0 & 1
    \end{pmatrix}
    -
    i
    \sum_{k=0}^{\infty}
    \frac{(-1)^k(at/2)^{2k+1}}{(2k+1)!}
    \begin{pmatrix}
      1 & 0 \\
      0 & -1
    \end{pmatrix}
    \nonumber
    \\
    &=
    \begin{pmatrix}
      e^{-iat/2} & 0 \\
      0 & e^{+iat/2}
    \end{pmatrix}
    \\
    \mathcal{H}_2
    &=
    -\frac{a_0}{2}
    \left\{  
      \begin{pmatrix}
        0 & \cos at \\
        \cos at & 0
      \end{pmatrix}
      -
      \begin{pmatrix}
        0 & -i\sin at \\
        i\sin at & 0
      \end{pmatrix}      
    \right\}
    \nonumber
    \\
    &=
    -\frac{a_0}{2}
    \begin{pmatrix}
      0 & e^{+ia t} \\
      e^{-iat} & 0
    \end{pmatrix}
  \end{align}
  となるので,微分方程式の係数は
  \begin{align}
    e^{+i\mathcal{H}_1t}
    \mathcal{H}_2
    e^{-i\mathcal{H}_1t}
    &=
    -\frac{a_0}{2}
    \begin{pmatrix}
      e^{-iat/2} & 0 \\
      0 & e^{+iat/2}
    \end{pmatrix}
    \begin{pmatrix}
      0 & e^{+iat} \\
      e^{-iat} & 0
    \end{pmatrix}
    \begin{pmatrix}
      e^{+iat/2} & 0 \\
      0 & e^{-iat/2}
    \end{pmatrix}
    \nonumber
    \\
    &=
    -\frac{a_0}{2}
    \begin{pmatrix}
      0 & 1 \\
      1 & 0
    \end{pmatrix}
  \end{align}
  となります.したがって,微分方程式は
  \begin{equation}
    i\pdv{}{t}
    \ket{\psi(t)}
    =
    -\frac{a_0}{2}
    \begin{pmatrix}
      0 & 1 \\
      1 & 0
    \end{pmatrix}
    \ket{\psi(t)}
  \end{equation}
  となりますが,これは設問4の方程式
  \begin{equation}    
    i\pdv{}{t}
    \ket{\phi(t)}
    =
    -\frac{\mu H}{2}
    \begin{pmatrix}
      0 & 1 \\
      1 & 0
    \end{pmatrix}
    \ket{\phi(t)}
  \end{equation}
  と同じです.(ただし,その設問では時間発展させて解きました.)初期条件も同じなので,解も一致します.よって,
  \begin{equation}
    \ket{\psi(t)}
    =
    \begin{pmatrix}
      \cos (a_0 t/2) \\
      i\sin (a_0 t/2)
    \end{pmatrix}    
  \end{equation}
  となるはずです.あとは,$\ket{\varphi(t)}=e^{-i\mathcal{H}_1t}\ket{\psi(t)}$という関係を用いれば
  \begin{equation}
    \mel*{\varphi(t)}{S_z}{\varphi(t)}
    =
    \mel*{\psi(t)}{(e^{+i\mathcal{H}_1t}S_ze^{-i\mathcal{H}_1t})}{\psi(t)}    
  \end{equation}
  となりますが,
  \begin{align}
    e^{+i\mathcal{H}_1t}S_ze^{-i\mathcal{H}_1t}
    &=
    \frac{1}{2}
    \begin{pmatrix}
      e^{-iat/2} & 0 \\
      0 & e^{+iat/2}
    \end{pmatrix}
    \begin{pmatrix}
      1 & 0 \\
      0 & -1
    \end{pmatrix}
    \begin{pmatrix}
      e^{+iat/2} & 0 \\
      0 & e^{-iat/2}
    \end{pmatrix}
    \nonumber
    \\
    &=
    \frac{1}{2}
    \begin{pmatrix}
      1 & 0 \\
      0 & -1
    \end{pmatrix}
  \end{align}
  なので,
  \begin{align}
    \mel*{\varphi(t)}{S_z}{\varphi(t)}
    &=
    \frac{1}{2}
    \begin{pmatrix}
      \cos (a_0t/2) & -i\sin (a_0t/2)
    \end{pmatrix}
    \begin{pmatrix}
      1 & 0 \\
      0 & -1
    \end{pmatrix}
    \begin{pmatrix}
      \cos (a_0t/2) \\
      i\sin (a_0t/2)
    \end{pmatrix}
    \nonumber
    \\
    &=
    \frac{1}{2}\cos a_0t
  \end{align}
  となります.


\end{enumerate}



\clearpage

\prb{2}{統計力学}

\begin{enumerate}

  \item 

  分配関数は,
  \begin{equation}
    Z[\beta]
    =
    \frac{1}{h^2}
    \int_{0}^{\pi}\dd \theta
    \int_{0}^{2\pi}\dd \varphi
    \int_{-\infty}^{\infty}\dd p_{\theta}
    \int_{-\infty}^{\infty}\dd p_{\varphi}\ 
    \exp
    \left[  
      -\frac{\beta}{2I}
      \left(  
        p_{\theta}^2
        +
        \frac{p_{\varphi}^2}{\sin^2\theta}
      \right)
    \right]
  \end{equation}
  なので,これを計算します\footnote{
    あとで知ったのですが,正準変数$q$とその共役な運動量$p$をとってくれば,次の積分測度
    \begin{equation*}
      \dd^3 \bm{r}\dd^3 \bm{p}
      =
      \prod_{l=1}^{3}
      \dd q_l\dd p_l
    \end{equation*}
    は不変なようです.だから,ヤコビアンなどは気にする必要がありません.(たぶん,ちゃんと考えると,$\dd^3\bm{r}$のヤコビアンと$\dd^3\bm{p}$のヤコビアンが相殺するのではないのでしょうか.勝手な想像ですが.)[参考]:\url{https://www.cmt.phys.kyushu-u.ac.jp/~A.Yoshimori/t12shkdans12.pdf}.
  }.$\varphi$の積分はそのまま行って$2\pi$.次に$p_\theta$と$p_{\varphi}$のガウス積分を行えば
  \begin{equation}
    Z[\beta]
    =
    \frac{4\pi^2Ik_BT}{h^2}
    \int_{0}^{\pi}
    |\sin\theta|\dd\theta
    =
    \frac{8\pi^2Ik_BT}{h^2}
  \end{equation}
  となります.よって,比熱は
  \begin{equation}
    U
    =
    k_B T
    ,\ 
    C_V
    =
    k_B
  \end{equation}
  となります.


  \item 

  分配関数は
  \begin{equation}
    Z[\beta]
    =
    \sum_{l=0}^{\infty}
    (2l+1)
    \exp\left[ -\frac{\beta\hbar^2}{2I}l(l+1) \right]
  \end{equation}
  です.また,高温では,$2l+1\sim 2l,\ l+1\sim l$と見なせて
  \begin{equation}
    Z[\beta]
    =
    \int_{0}^{\infty}
    2x\exp\left[ -\frac{\beta\hbar^2}{2I}x^2 \right]
    \dd x
  \end{equation}
  となるので,これを計算すれば
  \begin{equation}
    Z[\beta]
    =
    \frac{2I}{\hbar^2\beta}
  \end{equation}
  となります.

  
  \item 

  低温のときは,$l=0,\ 1$のみが効いてくると考えると
  \begin{equation}
    Z[\beta]
    \sim
    1
    +
    3\exp\left[ -\frac{\hbar^2\beta}{I} \right]
  \end{equation}
  なので,
  \begin{equation}
    U
    =
    \frac{3}{2}
    \frac{\hbar^2}{e^{\hbar^2/Ik_BT}+3}
    \sim
    \frac{3}{2}\hbar^2e^{-\hbar^2/Ik_BT}
  \end{equation}
  であり,これをさらに微分すれば,
  \begin{equation}
    C_V
    =
    \frac{3\hbar^4}{2k_BT}e^{-\hbar^2/Ik_BT}
  \end{equation}
  となります.


  \item 

  オルソ水素の原子核の入れ替えに対して波動関数は対称\footnote{
    合成されたスピンが1の状態は
    $$
      \ket{+,+}
      ,\ 
      \ket{+,-}+\ket{-,+}
      ,\ 
      \ket{-,-}
    $$
    ですが,これらはどれも対称です.(規格化はしてません.)
  },電子は入れ替えに対して反対称なので,回転による波動関数は反対称でないと,分子としては対称にならない.よって,オルソ水素なら$l$は奇数.同様に考えると,パラ水素については$l$は偶数.


  \item 

  縮退を考えれば,$N_0/N_P=3$だと考えられます.


  \item 

  分配関数は
  \begin{equation}
    Z[\beta]
    =
    Z_O^{3/4}[\beta]Z_P^{1/4}[\beta]
  \end{equation}
  と分離できるとすると,オルソ水素の分配関数は,$l=2k+1$のみをとって
  \begin{equation}
    Z_O[\beta]
    =
    \sum_{k=0}^{\infty}
    (4k+3)
    \exp\left[ -\beta k_B\Theta(2k+1)(2k+2) \right]
  \end{equation}
  となり,パラ水素のほうも同様に考えれば
  \begin{equation}
    Z_P[\beta]
    =
    \sum_{k=0}^{\infty}
    (4k+1)
    \exp\left[ -\beta k_B\Theta\cdot2k(2k+1) \right]
  \end{equation}
  となります.この状態で内部エネルギーを計算してみると
  \begin{align}
    U
    &=
    k_B\Theta
    \left[  
      \frac{3}{4}
      \frac{
        \sum_{k=0}^{\infty}
        (4k+3)(2k+1)(2k+2)
        \exp\left[ -\beta k_B\Theta(2k+1)(2k+2) \right]
        }{
        \sum_{k=0}^{\infty}
        (4k+3)
        \exp\left[ -\beta k_B\Theta(2k+1)(2k+2) \right]
        }
    \right.
    \nonumber
    \\
    &\left.\hspace{3cm}
        +
        \frac{1}{4}
        \frac{
          \sum_{k=0}^{\infty}
          (4k+1)\cdot2k(2k+1)
          \exp\left[ -\beta k_B\Theta\cdot2k(2k+1) \right]
          }{
          \sum_{k=0}^{\infty}
          (4k+1)
          \exp\left[ -\beta k_B\Theta\cdot2k(2k+1) \right]
          }
    \right]
  \end{align}
  となりますが,$T/\Theta=1/3$を低温だと思って,$k=0$の項のみで近似すれば
  \begin{equation}
    \frac{U}{k_B}
    \sim
    \Theta
    \times
    \frac{3}{4}
    \times
    \frac{6\cdot e^{-2\Theta}}{3\cdot e^{-2\Theta}}
    =
    135\ K
  \end{equation}
  です\footnote{
    これを低温だとみなすのはいかがなものかと思われるかもしれませんが,まあ,仕方ないんじゃないでしょうか.$k=1$を取り出すと,$\exp$を計算しないといけなくなります.
  }.


\end{enumerate}



\clearpage

\prb{3}{電磁気学}

\begin{enumerate}
  
  \item 

  復元力が$-m\omega^2\bm{r}$であることに注意すれば
  \begin{equation}
    m\dv[2]{\bm{r}}{t}
    =
    -
    e\bm{E}_0\cos\omega t
    -
    m\omega_0^2\bm{r}
  \end{equation}
  です.


  \item 

  $\bm{r}=\bm{R}\cos\omega t$を代入すれば
  \begin{equation}
    \bm{R}
    =
    \frac{e}{m(\omega^2-\omega_0^2)}\bm{E}_0
  \end{equation}
  となります.したがって,$\dot{\bm{r}}=-\omega\bm{R}\sin\omega t$より
  \begin{equation}
    \bm{i}_{\mathrm{d}}
    =
    -
    \frac{Ne^2\omega}{m(\omega^2-\omega^2_0)}\bm{E}_0\sin\omega t
  \end{equation}
  となります.


  \item 

  両辺の発散をとれば
  \begin{equation}
    \frac{1}{\mu_0}
    \bm{\nabla}\cdot(\bm{\nabla}\times\bm{B})
    =
    \varepsilon_0\pdv{}{t}\hspace{0pt}
    (\bm{\nabla}\cdot\bm{E})
    +
    \bm{\nabla}\cdot\bm{i}
  \end{equation}
  であり,ベクトル解析の公式から$\bm{\nabla}\cdot(\bm{\nabla}\times\bm{B})=0$なので,
  \begin{equation}
    \pdv{\rho}{t}
    +
    \bm{\nabla}\cdot\bm{i}
    =
    0
  \end{equation}
  となります.


  \item 

  式(1)に$\bm{i}_{\mathrm{d}}$を代入すれば
  \begin{equation}
    \frac{1}{\mu_0}\bm{\nabla}\times\bm{B}
    =
    -
    \left(  
      \varepsilon_0
      +
      \frac{Ne^2}{m(\omega^2-\omega^2_0)}
    \right)
    \omega\bm{E}_0\sin\omega t   
  \end{equation}
  なので,
  \begin{equation}
    C
    =
    \mu_0
    \left(  
      \varepsilon_0
      +
      \frac{Ne^2}{m(\omega^2-\omega^2_0)}
    \right)
  \end{equation}
  です.

  
  \item 

  前問の結果は
  \begin{equation}
    \varepsilon\mu
    =
    \mu_0
    \left(  
      \varepsilon_0
      +
      \frac{Ne^2}{m(\omega^2-\omega^2_0)}
    \right)
  \end{equation}
  となるので,
  \begin{equation}
    n(\omega)
    =
    \sqrt{
      1
      +
      \frac{Ne^2}{\varepsilon_0m(\omega^2-\omega^2_0)}      
    }
    =
    \sqrt{1+\frac{\omega_{\mathrm{p}}^2}{\omega^2-\omega^2_0}}
  \end{equation}
  です.


  \item 

  前問の結果より
  \begin{equation}
    n^2(\omega)
    =
    1+\frac{\omega_{\mathrm{p}}^2}{\omega^2-\omega_0^2}
  \end{equation}
  となるので,$\omega=\omega_0^2$となるところでグラフが変わります.よって,$\omega_0$の値によってスペクトルが変化して,図\ref{quartz},\ \ref{F}のようになります.

  \begin{figure}[ht]
      \centering
      \begin{tikzpicture}[scale=1.5]
        \draw[thin,-Stealth] (0,0)--(3.5,0)node[right]{$\omega$};
        \draw[thin,-Stealth] (0,0)--(0,3.5)node[left]{$n^2(\omega)$};        
        \draw[dotted,thin] (3.5,1.5)--(0,1.5)node[left]{1};
        \draw[dotted,thin] (1.2,3.5)--(1.2,0)node[below]{赤};
        \draw[dotted,thin] (2.5,3.5)--(2.5,0)node[below]{紫};
        \draw[thick,Stealth-Stealth] (1.2,3.5)--(2.5,3.5);
        \draw[dotted,thin] (3,3.5)--(3,0)node[below]{$\omega_0$};
        \begin{scope} \clip (0,0) rectangle (3.5,3.5);
          \draw[samples=100,domain=0:2.9,thick]plot(\x,{1.7-(2)/(3^2)+(2)/(-(\x)^2+3^2)}); 
          \draw[samples=100,domain=3.05:3.5,thick]plot(\x,{1.7-(2)/(3^2)+(2)/(-(\x)^2+3^2)});          
        \end{scope}
      \end{tikzpicture}
      \caption{石英の分散曲線}
      \label{quartz}
  \end{figure}

  \begin{figure}[ht]
    \centering
    \begin{tikzpicture}[scale=1.5]
      \draw[thin,-Stealth] (0,0)--(3.5,0)node[right]{$\omega$};
      \draw[thin,-Stealth] (0,0)--(0,3.5)node[left]{$n^2(\omega)$};   
      \draw[dotted,thin] (3.5,1.5)--(0,1.5)node[left]{1};
      \draw[dotted,thin] (1.2,3.5)--(1.2,0)node[below]{赤};
      \draw[dotted,thin] (2.5,3.5)--(2.5,0)node[below]{紫};
      \draw[thick,Stealth-Stealth] (1.2,3.5)--(2.5,3.5);
      \draw[dotted,thin] (1.7,3.5)--(1.7,0)node[below]{$\omega_0$};
      \begin{scope} \clip (0,0) rectangle (3.5,3.5);
        \draw[samples=100,domain=0:1.65,thick]plot(\x,{1.7-(1)/((1.7)^2)+(1)/(-(\x)^2+(1.7)^2)});        
        \draw[samples=100,domain=1.75:3.5,thick]plot(\x,{1.7-(1)/((1.7)^2)+(1)/(-(\x)^2+(1.7)^2)});           
      \end{scope}
    \end{tikzpicture}
    \caption{物質Fの分散曲線}
    \label{F}
  \end{figure}

\end{enumerate}

\end{document}
