\pdfoutput=1
\documentclass[a4paper,pdflatex,ja=standard]{bxjsarticle}

% ---Display \subsubsection at the Index
% \setcounter{tocdepth}{3}

% ---Setting about the geometry of the document----
% \usepackage{a4wide}
% \pagestyle{empty}

% ---Physics and Math Packages---
\usepackage{amssymb,amsfonts,amsthm,mathtools}
\usepackage{physics,braket,bm}

% ---underline---
\usepackage{ulem}

% ---cancel---
\usepackage{cancel}

% --- surround the texts or equations
\usepackage{fancybox,ascmac}

% ---settings of theorem environment---
\usepackage{amsthm}
\theoremstyle{definition}

% ---settings of proof environment---
\makeatletter
\renewenvironment{proof}[1][\proofname]{
  \par
  \pushQED{\qed}%
  % \normalfont\topsep6\p@\@plus6\p@\relax
  \trivlist
  \item\relax
  {#1}
  \hspace\labelsep\ignorespaces
}{%
  \popQED\endtrivlist\@endpefalse
}
\makeatother
\renewcommand{\proofname}{(解)}
\renewcommand{\qedsymbol}{\empty}

% ---Ignore the Warnings---
\usepackage{silence}
\WarningFilter{latexfont}{Some font shapes,Font shape}

% ---Insert the figure (If insert the `draft' at the option, the process becomes faster.)---
\usepackage{graphicx}
% \usepackage{subcaption}

% ----Add a link to a text---
\usepackage{url}
\usepackage{xcolor,hyperref}
\hypersetup{colorlinks=true,citecolor=orange,linkcolor=blue,urlcolor=magenta}
\usepackage[whole,autotilde]{bxcjkjatype}

% ---Tikz---
\usepackage{tikz,pgf,pgfplots,circuitikz}
\pgfplotsset{compat=1.15}
\usetikzlibrary{intersections,arrows.meta,angles,calc,3d,decorations.pathmorphing}

% ---Add the section number to the equation, figure, and table number---
\makeatletter
   \renewcommand{\theequation}{\thesection.\arabic{equation}}
   \@addtoreset{equation}{section}
   
   \renewcommand{\thefigure}{\thesection.\arabic{figure}}
   \@addtoreset{figure}{section}
   
   \renewcommand{\thetable}{\thesection.\arabic{table}}
   \@addtoreset{table}{section}
\makeatother

% ---surrond the text---
\usepackage{tcolorbox}
\tcbuselibrary{breakable,skins,theorems}
\definecolor{notecolor}{rgb}{0.2,0.4,0.8}
\definecolor{mondaicolor}{rgb}{0.6,0.6,0.2}
\newtcolorbox{note}[2][]
{
  enhanced,
  left=22pt,right=22pt,
  fonttitle=\bfseries,
  coltitle=white,
  colbacktitle=blue!50!black,
  attach boxed title to top left={},
  boxed title style={skin=enhancedfirst jigsaw,arc=1mm,bottom=0mm,boxrule=0mm},
  boxrule=0.5pt,
  colback=notecolor!5!,
  colframe=notecolor,
  sharp corners=northwest,
  drop fuzzy shadow,
  breakable,
  title=\vspace{3mm}#2,
  arc=1mm,
  #1
}
\newtcbtheorem{mondai}{演習問題}
{
  enhanced,
  left=22pt,right=22pt,
  fonttitle=\bfseries,
  coltitle=white,
  colbacktitle=green!70!black,
  attach boxed title to top left={},
  boxed title style={skin=enhancedfirst jigsaw,arc=1mm,bottom=0mm,boxrule=0mm},
  boxrule=0.5pt,
  colback=mondaicolor!5!,
  colframe=mondaicolor,
  sharp corners=northwest,
  drop fuzzy shadow,
  breakable,
  title=\vspace{3mm},
  arc=1mm
}{mondai}

% ---enumerate---
\renewcommand{\labelenumi}{$(\arabic{enumi})$}
% \renewcommand{\labelenumii}{$(\arabic{enumii})$}

% ---Index---
% \usepackage{makeidx}
% \makeindex 

% ---Fonts---
% \renewcommand{\familydefault}{\sfdefault}

% ---Title---
\title{速習ノート}
\author{\empty}
\date{最終更新:\today}

\begin{document}

\maketitle

\tableofcontents
\clearpage

\section{微分方程式}

今回の微分方程式は
\begin{equation}
  m\dv[2]{x}{t}
  =
  -kx
  +
  qE_0\sin\omega t
\end{equation}
でした.両辺を$m$で割って
\begin{equation}
  \dv[2]{x}{t}
  =
  -\frac{k}{m}x
  +
  \frac{qE_0}{m}\sin\omega t  
  \label{diff_eq}
\end{equation}
としておきます.こういう場合,特解(振動数$\omega$の解)は
\begin{equation}
  x(t)
  =
  A\sin \omega t
\end{equation}
とおいて探します\footnote{
  定数変化法でも見つけられないわけではないと思うけど,あまり見たことがない気がする.多分こっちのほうがラクだから.
}.\eqref{diff_eq}に代入すると
\begin{equation}
  -A\omega^2\sin\omega t
  =
  -\frac{k}{m}A\sin\omega t
  +
  \frac{qE_0}{m}\sin\omega t
  \quad
  \rightarrow
  \quad
  A
  =
  \frac{qE_0/m}{(k/m)-\omega^2}
\end{equation}
となるので,特解は
\begin{equation}
  x(t)
  =
  \frac{qE_0/m}{(k/m)-\omega^2}\sin\omega t
\end{equation}
です.ここで,$\omega_0\equiv\sqrt{k/m}$が共振振動数で,$\omega\sim\omega_0$で$x(t)$がすごい振動します.
\\

こんな感じで特解を出します.ちょっと院試に出そうな問題をピックアップしておきます.

\begin{mondai}{非斉次微分方程式}{inhom}
  次の微分方程式の特解を求めよ.
  \begin{enumerate}
    \item 
    ${\displaystyle
    \dv[2]{x}{t}+2\dv{x}{t}-3x=e^{2t}}$
    \footnote{
      参考:\url{https://www.geisya.or.jp/~mwm48961/kou3/differ_eq3.htm}
    }
    \item 
    ${
      \displaystyle
      \dv[2]{x}{t}+2\gamma\dv{x}{t}+\omega_0^2x
      =
      f\sin\omega t
    }$
    \footnote{
      参考:\url{http://www.asem.kyushu-u.ac.jp/qq/qq02/kikanbuturi/chap6.pdf}
    }
    \ (減衰振動.今日のよりもちょっと難しい.)
  \end{enumerate}
\end{mondai}

\uline{解答}
\begin{enumerate}
  \item 
  こういうのは$x(t)=Ae^{2t}$として$A$を決めます.代入すると$A=1/5$となるので,特解は
  \begin{equation}
    x(t)
    =
    \frac{1}{5}e^{2t}
  \end{equation}
  です.(計算してみると,なんでこのようにおけばよいのか分かるかも.)
  \item 
  ワンチャン力学で出るかもと思って出しておきました.知ってたらごめん.今回の東大のケースと同じように$x(t)=A\sin\omega t$とおけばうまくいくと思いきや,これだとダメです.今回は
  \begin{equation}
    x(t)
    =
    A\cos\omega t
    +
    B\sin\omega t
  \end{equation}
  として,$A,B$を求めます.計算は省略しますが,答えは
  \begin{equation}
    A
    =
    \frac{-2\gamma\omega f}{(\omega_0^2-\omega^2)^2+(2\gamma\omega)^2}
    ,\ 
    B
    =
    \frac{(\omega_0^2-\omega^2)f}{(\omega_0^2-\omega^2)^2+(2\gamma\omega)^2}
  \end{equation}
  です\footnote{
    実は複素数でやるとチョットだけラクになるけど,こっちのほうが直感的で嬉しい.
  }.
\end{enumerate}


\section{量子力学の復習}

今日,電車で言ってた話.
\begin{mondai}{波動関数の問題}{wave_function}
  \begin{enumerate}
    \item 
    $\ev*{\bm{p}|\bm{q}}$を求めよ.ただし,規格化はしなくてよい.
    \item 
    今回の問題を復習する.つまり,$\psi(q,t)\equiv\ev*{q|\psi(t)}$を求めよ.ただし,前問で
    \begin{equation}
      \hat{p}
      \ket{(r,\phi)}
      =
      \alpha(r,\phi)\hat{q}
      \ket{(r,\phi)}
      ,\ 
      \ket{\psi(t)}
      =
      \ket{(t,-\pi/4)}
    \end{equation}
    を示していたので,これは用いてもよい.
  \end{enumerate}
\end{mondai}

\uline{解答}
\begin{enumerate}
  \item 
  こういうのは大体$\ev*{\bm{p}|\hat{\bm{p}}|\bm{q}}$を計算する.$\hat{\bm{p}}$が$\ket{\bm{q}}$に作用すると思えば
  \begin{equation}
    \ev*{\bm{p}|\hat{\bm{p}}|\bm{q}}
    =
    -i\hbar\pdv{}{\bm{q}}\ev*{\bm{p}|\bm{q}}
    \label{eqn1}
  \end{equation}
  となり,一方で$\bra{\bm{p}}$に$\hat{\bm{p}}$が作用すると思えば
  \begin{equation}
    \ev*{\bm{p}|\hat{\bm{p}}|\bm{q}}
    =
    \bm{p}\ev*{\bm{p}|\bm{q}}
    \label{eqn2}
  \end{equation}
  です.よって,\eqref{eqn1}と\eqref{eqn2}を合わせれば
  \begin{equation}
    -i\hbar\pdv{}{\bm{q}}\ev*{\bm{p}|\bm{q}}    
    =
    \bm{p}\ev*{\bm{p}|\bm{q}}
  \end{equation}
  という微分方程式になりますが,これは
  \begin{equation}
    \frac{1}{\ev*{\bm{p}|\bm{q}}}\pdv{}{\bm{q}}\ev*{\bm{p}|\bm{q}}
    =
    \frac{i}{\hbar}\bm{p}
  \end{equation}
  より
  \begin{equation}
    \ev*{\bm{p}|\bm{q}}
    =
    e^{i\bm{q}\cdot\bm{p}/\hbar}
  \end{equation}
  と解けます\footnote{
    「$\bm{q}$の微分って?」と思ったら,とりあえず1次元で考えると分かりやすいかも?
  }.
  \item 
  おんなじ感じです.$r=t,\phi=-\pi/4$であることに注意すれば
  \begin{equation}
    \ev*{q|\hat{p}|(t,-\pi/4)}
  \end{equation}
  を2通りで計算すればいいです.$\hat{p}$が$\bra{q}$に作用すると思えば,
  \begin{equation}
    \ev*{q|\hat{p}|(t,-\pi/4)}
    =
    -i\hbar\pdv{}{q}\psi(q,t)
  \end{equation}  
  で,$\hat{p}$が$\ket{(t,-\pi/4)}$に作用すると思えば,前問で求めた公式を用いて,
  \begin{equation}
    \ev*{q|\hat{p}|(t,-\pi/4)}
    =
    \alpha(t,-\pi/4)
    \ev*{q|\hat{q}|(t,-\pi/4)}
    =
    \alpha(t,-\pi/4)q\psi(q,t)
  \end{equation}
  となります.よって,微分方程式は
  \begin{equation}
    -i\hbar\pdv{}{q}\psi(q,t)
    =    
    \alpha(t,-\pi/4)q\psi(q,t)
  \end{equation}
  なので,これを解くと
  \begin{equation}
    \psi(q,t)
    =
    \exp
    \left[  
      i\frac{\alpha(t,-\pi/4) q^2}{2\hbar}
    \right]
  \end{equation}
  がたぶん答えです.
\end{enumerate}

\vspace{10pt}

俺は先に夏休み満喫してます.あと少し,がんばって!!

\end{document}
