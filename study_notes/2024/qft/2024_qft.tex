\documentclass[unicode,a4paper,11pt]{ltjsarticle}

\usepackage{luatexja-fontspec}
\setmainfont{TeX Gyre Termes}
\setmainjfont[BoldFont = IPAGothic]{IPAMincho}
\setmathrm{Latin Modern Roman}
% \setmainjfont{Noto Sans JP}

% ---Display \subsubsection at the Index
% \setcounter{tocdepth}{3}

% ---Setting about the geometry of the document----
% \usepackage{a4wide}
% \pagestyle{empty}

% ---Physics and Math Packages---
\usepackage{amssymb,amsfonts,amsthm,mathtools}
\usepackage{physics,braket,bm}

% ---underline---
\usepackage[normalem]{ulem}

% ---cancel---
\usepackage{cancel}

% --- surround the texts or equations
% \usepackage{fancybox,ascmac}

% ---settings of theorem environment---
% \usepackage{amsthm}
% \theoremstyle{definition}

% ---settings of proof environment---
% \renewcommand{\proofname}{\textbf{証明}}
% \renewcommand{\qedsymbol}{$\blacksquare$}

% ---Ignore the Warnings---
\usepackage{silence}
\WarningFilter{latexfont}{Some font shapes}
\WarningFilter{latexfont}{Font shape}
\WarningFilter{latexfont}{Size substitutions}
\ExplSyntaxOn
\msg_redirect_name:nnn{hooks}{generic-deprecated}{none}
\ExplSyntaxOff

% ---Insert the figure (If insert the `draft' at the option, the process becomes faster.)---
\usepackage{graphicx}
% \usepackage{subcaption}

% ----Add a link to a text---
\usepackage{url,hyperref}
\usepackage[dvipsnames,svgnames]{xcolor}
\hypersetup{colorlinks=true,citecolor=FireBrick,linkcolor=Navy,urlcolor=purple}
% ---refer `texdoc xcolor' at the command line---

% ---Tikz---
% \usepackage{tikz,pgf,pgfplots,circuitikz}
% \pgfplotsset{compat=1.15}
% \usetikzlibrary{intersections,arrows.meta,angles,calc,3d,decorations.pathmorphing}

% ---Add the section number to the equation, figure, and table number---
\makeatletter
   \renewcommand{\theequation}{\thesection.\arabic{equation}}
   \@addtoreset{equation}{section}
   
   \renewcommand{\thefigure}{\thesection.\arabic{figure}}
   \@addtoreset{figure}{section}
   
   \renewcommand{\thetable}{\thesection.\arabic{table}}
   \@addtoreset{table}{section}
\makeatother

% ---enumerate---
% \renewcommand{\labelenumi}{$\arabic{enumi}.$}
% \renewcommand{\labelenumii}{$(\arabic{enumii})$}

% ---Index---
% \usepackage{makeidx}
% \makeindex 

% ---Fonts---
% \renewcommand{\familydefault}{\sfdefault}

% ---Title---
\title{
  場の量子論
}
\author{
  宮根\ 一樹
}
\date{\today}

\begin{document}

\maketitle
\tableofcontents

\clearpage
\section{くりこみと対称性}



















% ----------------------------------------

\clearpage

\makeatletter
\renewcommand{\appendix}{\par
  \setcounter{section}{0}%
  \setcounter{subsection}{0}%
  \gdef\presectionname{\appendixname}%
  \gdef\postsectionname{}%
  \gdef\thesection{\presectionname\@Alph\c@section\postsectionname}%
  \gdef\thesubsection{\@Alph\c@section.\@arabic\c@subsection}%
  \renewcommand{\theequation}{\@Alph\c@section.\arabic{equation}}%
  \renewcommand{\thefigure}{\@Alph\c@section.\arabic{figure}}%
  \renewcommand{\thetable}{\@Alph\c@section.\arabic{table}}%
}
\makeatother
\appendix

\section{マクスウェル理論のラグランジアンの規格化}

マクスウェルのラグランジアンは
\begin{equation}
  \mathcal{L}
  =
  NF^{\mu\nu}F_{\mu\nu}
\end{equation}
と書ける。ただし、場の強度は$F^{\mu\nu}=\partial^{\mu}A^{\nu}-\partial^{\nu}A^{\mu}$である。ここでは、係数$N$を決定したい。そのためには運動項が
\begin{equation}
  \mathcal{L}
  =
  \frac{1}{2}\dot{A}_{1}^2
  +
  \frac{1}{2}\dot{A}_{2}^2
  +
  \frac{1}{2}\dot{A}_{3}^2
  +
  \cdots
\end{equation}
となっていればよくて、$F^{\mu\nu}F_{\mu\nu}$を丁寧に展開していけば
\begin{align}
  F^{\mu\nu}F_{\mu\nu}
   & =
  (\partial^{\mu}A^{\nu}-\partial^{\nu}A^{\mu})
  (\partial_{\mu}A_{\nu}-\partial_{\nu}A_{\mu})
  \nonumber
  \\
   & =
  2((\partial^{\mu}A^{\nu})(\partial_{\mu}A_{\nu})-(\partial_{\mu}A^{\nu})(\partial_{\nu}A_{\mu}))
  \nonumber
  \\
   & =
  -2(\dot{A}_{1}^2+\dot{A}_{2}^2+\dot{A}_{3}^2)+\cdots
\end{align}
となり、$N$は$-2N=1/2$より$N=-1/4$である。したがって、電磁場のラグランジアンは
\begin{equation}
  \mathcal{L}
  =
  -\frac{1}{4}F^{\mu\nu}F_{\mu\nu}
\end{equation}
となる。







% ----------------------------------------
\clearpage
\bibliography{ref}
\bibliographystyle{ytamsalpha}

\nocite{Peskin:1995}
\nocite{Fujii:2005}

\end{document}
