\documentclass[unicode,a4paper,11pt]{ltjsarticle}

\usepackage{luatexja-fontspec}
\setmainfont{TeX Gyre Termes}
\setmainjfont[BoldFont = IPAGothic]{IPAMincho}
\setmathrm{Latin Modern Roman}
% \setmainjfont{Noto Sans JP}

% ---Display \subsubsection at the Index
% \setcounter{tocdepth}{3}

% ---Setting about the geometry of the document----
% \usepackage{a4wide}
% \pagestyle{empty}

% ---Physics and Math Packages---
\usepackage{amssymb,amsfonts,amsthm,mathtools}
\usepackage{physics,braket,bm}

% ---underline---
\usepackage[normalem]{ulem}

% ---cancel---
\usepackage{cancel}

% --- surround the texts or equations
% \usepackage{fancybox,ascmac}

% ---settings of theorem environment---
\theoremstyle{definition}
\newtheorem{dfn}{定義}[section]
\newtheorem{prop}{命題}[section]
\newtheorem{thm}{定理}[section]
\newtheorem{exm}{例}[section]
\newtheorem{exc}{演習}[section]

% ---settings of proof environment---
\renewcommand{\proofname}{\textbf{証明}}
\renewcommand{\qedsymbol}{$\blacksquare$}

% ---Ignore the Warnings---
\usepackage{silence}
\WarningFilter{latexfont}{Some font shapes}
\WarningFilter{latexfont}{Font shape}
\WarningFilter{latexfont}{Size substitutions}
\ExplSyntaxOn
\msg_redirect_name:nnn{hooks}{generic-deprecated}{none}
\ExplSyntaxOff

% ---Insert the figure (If insert the `draft' at the option, the process becomes faster.)---
\usepackage{graphicx}
% \usepackage{subcaption}

% ----Add a link to a text---
\usepackage{url,hyperref}
\usepackage[dvipsnames,svgnames]{xcolor}
\hypersetup{colorlinks=true,citecolor=FireBrick,linkcolor=Navy,urlcolor=purple}
% ---refer `texdoc xcolor' at the command line---

% ---Tikz---
% \usepackage{tikz,pgf,pgfplots,circuitikz}
% \pgfplotsset{compat=1.15}
% \usetikzlibrary{intersections,arrows.meta,angles,calc,3d,decorations.pathmorphing}

% ---Add the section number to the equation, figure, and table number---
\makeatletter
   \renewcommand{\theequation}{$\thesection.\arabic{equation}$}
   \@addtoreset{equation}{section}
   
   \renewcommand{\thefigure}{\thesection.\arabic{figure}}
   \@addtoreset{figure}{section}
   
   \renewcommand{\thetable}{\thesection.\arabic{table}}
   \@addtoreset{table}{section}
\makeatother

% ---enumerate---
% \renewcommand{\labelenumi}{$\arabic{enumi}.$}
% \renewcommand{\labelenumii}{$(\arabic{enumii})$}

% ---Index---
% \usepackage{makeidx}
% \makeindex 

% ---Fonts---
% \renewcommand{\familydefault}{\sfdefault}

% ---Title---
\title{
  ゲージ理論と幾何
}
\author{
  宮根\ 一樹
}
\date{\today}

\begin{document}

\maketitle
\tableofcontents

\clearpage
\section{多様体とその周辺}

\subsection{多様体}

まずは多様体の定義から。

\begin{dfn}
  ハウスドルフ空間$M$に対して、$M$が開集合$U_{i}$によって
  \begin{equation}
    U
    =
    \bigcup_{i}U_{i}
  \end{equation}
  で表され、各$U_{i}$に対して、$U_{i}$から$n$次元ベクトル空間$\mathbb{R}^{n}$への全単射$\bm{x}_{U_{i}}$があって
  \begin{itemize}
    \item
          像$\bm{x}_{U_{i}}(U)$は$\mathbb{R}^{n}$の開集合で、$\bm{x}_{U_{i}}$は$U_{i}$から$\bm{x}_{U_{i}}(U)$への同相写像。
    \item
          $U_{i}\cap U_{j}\neq\phi$ならば、写像
          \begin{equation}
            \bm{x}_{U_{j}}\circ\bm{x}_{U_{i}}^{-1}
            :
            \bm{x}_{U_{i}}(U_{i}\cap U_{j})\rightarrow\bm{x}_{U_{j}}(U_{i}\cap U_{j})
          \end{equation}
          が全単射で$C^{\infty}$かつ逆写像も同様。
  \end{itemize}
  を満たすとき、
  \begin{itemize}
    \item
          組$\{(U_{i},\bm{x}_{U_{i}})\}$の全体は$M$に$C^{\infty}$構造を与え、
    \item
          $M$を$C^{\infty}$多様体という。
  \end{itemize}
\end{dfn}

\begin{exm}[直積多様体]
  $m$次元$C^{\infty}$多様体$M$、$n$次元$C^{\infty}$多様体$N$の$C^{\infty}$構造が$\{(U_{\alpha},\phi_{\alpha})\}_{\alpha\in A},\{(V_{\beta},\psi_{\beta})\}_{\beta\in B}$で定められているとき、
  \begin{equation}
    (\phi_{\alpha}\times\psi_{\beta})(x,y)
    =
    (\phi_{\alpha}(x),\psi_{\beta}(y))
  \end{equation}
  で定義しておけば、$\{(U_{\alpha}\times V_{\beta},\phi_{\alpha}\times\psi_{\beta})\}_{(\alpha,\beta)\in A\times B}$は$M\times N$上に$C^{\infty}$構造が定められて、$M\times N$は$C^{\infty}$多様体になる。これは直積多様体。
\end{exm}

\begin{exm}[$n$次元球面]
  $\mathbb{R}^{n+1}$に対して
  \begin{equation}
    S^{n}
    =
    \{
    (x^{1},\cdots,x^{n+1})\in\mathbb{R}^{n+1}
    |
    \sum (x^{i})^2=1
    \}
  \end{equation}
  において、各$i=1,\cdots,n+1$に対して
  \begin{equation}
    U_{i}^{\pm}
    \equiv
    \{
    (x^{1},\cdots,x^{i+1})\in S^{n}
    |
    x^{i}\gtrless 0
    \}
  \end{equation}
  とおいて、$x_{i}^{\pm}:U_{i}^{\pm}\rightarrow\mathbb{R}^{n}$を
  \begin{equation}
    x_{i}^{\pm}(x^{1},\cdots,x^{n+1})
    =
    (x^{1},\cdots,\hat{x^{i}},\cdots,x^{n+1})
    \in \mathbb{R}^{n}
  \end{equation}
  とする\footnote{
    ハット$\hat{\ }$はその成分がないことを表す。
  }と、$x^{\pm}_{i}$の像は$\mathbb{R}^{n}$の開球(内部を含む)である。これらは全単射で、$\{(U_{i}^{\pm},x_{i}^{\pm})\}_{i=1,\cdots,n}$が$S^{n}$上に$C^{\infty}$構造を定めるため、$S^{n}$は多様体。
\end{exm}

\begin{exm}[射影空間]
  $\mathbb{K}=\mathbb{R}\ \mathrm{or}\ \mathbb{C}$に対して、
\end{exm}
















































% ----------------------------------------
% \clearpage

% \makeatletter
% \renewcommand{\appendix}{\par
%   \setcounter{section}{0}%
%   \setcounter{subsection}{0}%
%   \gdef\presectionname{\appendixname}%
%   \gdef\postsectionname{}%
%   \gdef\thesection{\presectionname\@Alph\c@section\postsectionname}%
%   \gdef\thesubsection{\@Alph\c@section.\@arabic\c@subsection}%
%   \renewcommand{\theequation}{\@Alph\c@section.\arabic{equation}}%
%   \renewcommand{\thefigure}{\@Alph\c@section.\arabic{figure}}%
%   \renewcommand{\thetable}{\@Alph\c@section.\arabic{table}}%
% }
% \makeatother
% \appendix

% \section{***}



% ----------------------------------------
\clearpage
\bibliography{ref}
\bibliographystyle{unsrt}

\nocite{Mogi:2001}
\nocite{Nakahara:2003}

\end{document}
